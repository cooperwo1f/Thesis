%%%%%%%%%%%%%%%%%%%%%%%%%%%%%%%%%%%%%%%%%
% Masters/Doctoral Thesis
% LaTeX Template
% Version 2.5 (27/8/17)
%
% This template was downloaded from:
% http://www.LaTeXTemplates.com
%
% Version 2.x major modifications by:
% Vel (vel@latextemplates.com)
%
% This template is based on a template by:
% Steve Gunn (http://users.ecs.soton.ac.uk/srg/softwaretools/document/templates/)
% Sunil Patel (http://www.sunilpatel.co.uk/thesis-template/)
%
% Template license:
% CC BY-NC-SA 3.0 (http://creativecommons.org/licenses/by-nc-sa/3.0/)
%
%%%%%%%%%%%%%%%%%%%%%%%%%%%%%%%%%%%%%%%%%

%----------------------------------------------------------------------------------------
%	PACKAGES AND OTHER DOCUMENT CONFIGURATIONS
%----------------------------------------------------------------------------------------

\documentclass[
12pt, % The default document font size, options: 10pt, 11pt, 12pt
oneside, % Two side (alternating margins) for binding by default, uncomment to switch to one side
english, % ngerman for German
onehalfspacing, % Single line spacing, alternatives: onehalfspacing or doublespacing
%draft, % Uncomment to enable draft mode (no pictures, no links, overfull hboxes indicated)
%nolistspacing, % If the document is onehalfspacing or doublespacing, uncomment this to set spacing in lists to single
%liststotoc, % Uncomment to add the list of figures/tables/etc to the table of contents
%toctotoc, % Uncomment to add the main table of contents to the table of contents
%parskip, % Uncomment to add space between paragraphs
%nohyperref, % Uncomment to not load the hyperref package
headsepline, % Uncomment to get a line under the header
%chapterinoneline, % Uncomment to place the chapter title next to the number on one line
%consistentlayout, % Uncomment to change the layout of the declaration, abstract and acknowledgements pages to match the default layout
]{MastersDoctoralThesis} % The class file specifying the document structure

\usepackage[utf8]{inputenc} % Required for inputting international characters
\usepackage[T1]{fontenc} % Output font encoding for international characters
\usepackage{todonotes}
\usepackage{mathpazo} % Use the Palatino font by default

%\usepackage[style=numeric]{biblatex} % Use the bibtex backend with the authoryear citation style (which resembles APA)
\usepackage[backend=bibtex,style=numeric]{biblatex}
\bibliography{thesis, chapters/literature_review/literature_review, chapters/project_plan/project_plan}
\nocite{*}

\usepackage[autostyle=true]{csquotes} % Required to generate language-dependent quotes in the bibliography
\usepackage{multirow}
\usepackage{colortbl}
\usepackage{listings}
\usepackage{titlesec}
\titleformat{\chapter}{\huge\raggedright}{\MakeUppercase{\thechapter}}{1em}{}

\lstset{basicstyle=\ttfamily\footnotesize,breaklines=true}

%----------------------------------------------------------------------------------------
%	MARGIN SETTINGS
%----------------------------------------------------------------------------------------

\geometry{
	paper=a4paper, % Change to letterpaper for US letter
	inner=2.5cm, % Inner margin
	outer=3.8cm, % Outer margin
	bindingoffset=.5cm, % Binding offset
	top=1.5cm, % Top margin
	bottom=1.5cm, % Bottom margin
	%showframe, % Uncomment to show how the type block is set on the page
}

%----------------------------------------------------------------------------------------
%	THESIS INFORMATION
%----------------------------------------------------------------------------------------

% Your thesis title, this is used in the title and abstract, print it elsewhere with \ttitle
\thesistitle{Performance Augmentation using Biosignals}
% Your supervisor's name, this is used in the title page, print it elsewhere with \supname
\supervisor{Associate Professor Kenneth Pope}
% Your examiner's name, this is not currently used anywhere in the template, print it elsewhere with \examname
\examiner{}
% Your degree name, this is used in the title page and abstract, print it elsewhere with \degreename
\degree{Bachelor of Engineering (Electronics) (Honours)}
% Your name, this is used in the title page and abstract, print it elsewhere with \authorname
\author{Cooper Wolfden}
% Your address, this is not currently used anywhere in the template, print it elsewhere with \addressname
\addresses{}

% Your subject area, this is not currently used anywhere in the template, print it elsewhere with \subjectname
\subject{Electronic Engineering}
% Keywords for your thesis, this is not currently used anywhere in the template, print it elsewhere with \keywordnames
\keywords{Biosensors, Lighting control, Musical instrument digital interfaces, Wearable sensors}
% Your university's name and URL, this is used in the title page and abstract, print it elsewhere with \univname
\university{\href{https://www.flinders.edu.au/}{Flinders University}}
% Your department's name and URL, this is used in the title page and abstract, print it elsewhere with \deptname
\department{\href{https://www.flinders.edu.au/college-science-engineering}{College of Science and Engineering}}
% Your research group's name and URL, this is used in the title page, print it elsewhere with \groupname
\group{}
% Your faculty's name and URL, this is used in the title page and abstract, print it elsewhere with \facname
\faculty{\href{}{}}

\AtBeginDocument{
\hypersetup{pdftitle=\ttitle} % Set the PDF's title to your title
\hypersetup{pdfauthor=\authorname} % Set the PDF's author to your name
\hypersetup{pdfkeywords=\keywordnames} % Set the PDF's keywords to your keywords
}

\begin{document}

\frontmatter % Use roman page numbering style (i, ii, iii, iv...) for the pre-content pages

\pagestyle{plain} % Default to the plain heading style until the thesis style is called for the body content

%----------------------------------------------------------------------------------------
%	TITLE PAGE
%----------------------------------------------------------------------------------------

\begin{titlepage}
\begin{center}

\vspace*{.06\textheight}
{\scshape\LARGE \univname\par}\vspace{1.5cm} % University name
\textsc{\Large Honours Thesis}\\[0.5cm] % Thesis type

\HRule\\[0.4cm] % Horizontal line
{\huge \bfseries \ttitle\par}\vspace{0.4cm} % Thesis title
\HRule\\[1.5cm] % Horizontal line

\begin{minipage}[t]{0.4\textwidth}
\begin{flushleft} \large
\emph{Author:}\\
{\authorname} % Author name - remove the \href bracket to remove the link
\end{flushleft}
\end{minipage}
\begin{minipage}[t]{0.4\textwidth}
\begin{flushright} \large
\emph{Supervisor:} \\
{\supname} % Supervisor name - remove the \href bracket to remove the link
\end{flushright}
\end{minipage}\\[3cm]

\vfill

\large \textit{A thesis submitted in fulfillment of the requirements\\ for the degree of \degreename}\\[0.3cm] % University requirement text
%%\groupname\\\deptname\\[2cm] % Research group name and department name

\vfill

{\large October 16, 2023}\\[4cm] % Date
%\includegraphics{Logo} % University/department logo - uncomment to place it

\vfill
\end{center}
\end{titlepage}

%----------------------------------------------------------------------------------------
%	DECLARATION PAGE
%----------------------------------------------------------------------------------------

\begin{declaration}
\addchaptertocentry{\authorshipname} % Add the declaration to the table of contents
\noindent I, \authorname, declare that this thesis titled, \enquote{\ttitle} and the work presented in it are my own. I confirm that:

\begin{itemize}
\item This work was done wholly while in candidature for a degree of \degreename.
\item This document is in accordance with the plagiarism policy of \univname.
\item Where any part of this thesis has previously been submitted for a degree or any other qualification at this University or any other institution, this has been clearly stated.
\item Where I have consulted the published work of others, this is always clearly attributed.
\item Where I have quoted from the work of others, the source is always given. With the exception of such quotations, this thesis is entirely my own work.
\item I have acknowledged all main sources of help.
\item Where the thesis is based on work done by myself jointly with others, I have made clear exactly what was done by others and what I have contributed myself.\\
\end{itemize}

\noindent Signed:\\
\rule[0.5em]{25em}{0.5pt} % This prints a line for the signature

\noindent Date:\\
\rule[0.5em]{25em}{0.5pt} % This prints a line to write the date
\end{declaration}

\cleardoublepage

%----------------------------------------------------------------------------------------
%	QUOTATION PAGE
%----------------------------------------------------------------------------------------

%\vspace*{0.2\textheight}

%\noindent\enquote{\itshape One of the major problems encountered in time travel is not that of becoming your own father or mother. There is no problem in becoming your own father or mother that a broad-minded and well-adjusted family can't cope with. There is no problem with changing the course of history—the course of history does not change because it all fits together like a jigsaw. All the important changes have happened before the things they were supposed to change and it all sorts itself out in the end.

%The major problem is simply one of grammar, and the main work to consult in this matter is Dr. Dan Streetmentioner's Time Traveler's Handbook of 1001 Tense Formations. It will tell you, for instance, how to describe something that was about to happen to you in the past before you avoided it by time-jumping forward two days in order to avoid it. The event will be descibed differently according to whether you are talking about it from the standpoint of your own natural time, from a time in the further future, or a time in the further past and is futher complicated by the possibility of conducting conversations while you are actually traveling from one time to another with the intention of becoming your own mother or father.

%Most readers get as far as the Future Semiconditionally Modified Subinverted Plagal Past Subjunctive Intentional before giving up; and in fact in later aditions of the book all pages beyond this point have been left blank to save on printing costs.

%The Hitchhiker's Guide to the Galaxy skips lightly over this tangle of academic abstraction, pausing only to note that the term "Future Perfect" has been abandoned since it was discovered not to be.}\bigbreak

%\hfill The Hitch Hiker's Guide To The Galaxy

%----------------------------------------------------------------------------------------
%	ABSTRACT PAGE
%----------------------------------------------------------------------------------------

\begin{abstract}
  \addchaptertocentry{\abstractname} % Add the abstract to the table of contents

  This thesis aims to establish a platform for performers to utilize biological signals in order to create engaging live performances.
  This project is a continuation of the `Music from Biosignals' project, with the new addition of lighting control.

  The initial phase of the project began with the implementation of the lighting controller,
  enabling the system to interface with standard performance lighting fixtures.
  There are two parts of the lighting controller,
  an off-the-shelf controller, ensuring high reliability,
  and a prototyping fixture, allowing fast and cost-effective testing.
  These additions allow for the creation of exciting and engaging lighting patterns from collected biosensors.

  Further project developments involved the previously designed hardware,
  consisting of a PIC32 microcontroller, an ESP32 wireless transmitter, and an ADS1294R analog-to-digital converter.
  For this hardware, programming inconsistencies were resolved,
  over-the-air updates were implemented for the ESP32,
  and hardware debugging was enabled for the PIC32.

  Following this, consistent device-to-device communication was etablished.
  With successful communication between the PIC32 and ESP32 over SPI,
  challenges overcome with communication between the ADS1294R and PIC32,
  and the ESP32 acting as a wireless bridge with the off-body PC via TCP/IP.

  Despite these achievements, power issues during the final stage of development caused the device to fail in providing wireless sensor readings for the off-body PC to process.
  Thorough testing showed that the issue was related to the high impedance of the power plane.
  In conclusion, while the performance augmentation platform is not fully functional, significant strides have been made in resolving communication issues and establishing a foundation for future iterations.
  Recommendations for a second version include addressing power distribution challenges and optimizing overall current draw.
\end{abstract}

%----------------------------------------------------------------------------------------
%	ACKNOWLEDGEMENTS
%----------------------------------------------------------------------------------------

\begin{acknowledgements}
\addchaptertocentry{\acknowledgementname} % Add the acknowledgements to the table of contents
I would like to thank Craig Dawson for their professional advice and skills that helped immensely in the development of this project.
I would like to thank Associate Professor Kenneth Pope for their help in guiding me throughout this project and providing the much needed pressure when necessary.
Lastly, I would like to thank my friends and family for providing much needed relief from the various stresses of this year.
\end{acknowledgements}

%----------------------------------------------------------------------------------------
%	LIST OF CONTENTS/FIGURES/TABLES PAGES
%----------------------------------------------------------------------------------------

\tableofcontents % Prints the main table of contents

\listoffigures % Prints the list of figures

%\listoftables % Prints the list of tables

%----------------------------------------------------------------------------------------
%	ABBREVIATIONS
%----------------------------------------------------------------------------------------

%\begin{abbreviations}{ll} % Include a list of abbreviations (a table of two columns)

%\textbf{LAH} & \textbf{L}ist \textbf{A}bbreviations \textbf{H}ere\\
%\textbf{WSF} & \textbf{W}hat (it) \textbf{S}tands \textbf{F}or\\

%\end{abbreviations}

%----------------------------------------------------------------------------------------
%	PHYSICAL CONSTANTS/OTHER DEFINITIONS
%----------------------------------------------------------------------------------------

%\begin{constants}{lr@{${}={}$}l} % The list of physical constants is a three column table

% The \SI{}{} command is provided by the siunitx package, see its documentation for instructions on how to use it

%Speed of Light & $c_{0}$ & \SI{2.99792458e8}{\meter\per\second} (exact)\\
%Constant Name & $Symbol$ & $Constant Value$ with units\\

%\end{constants}

%----------------------------------------------------------------------------------------
%	SYMBOLS
%----------------------------------------------------------------------------------------

%\begin{symbols}{lll} % Include a list of Symbols (a three column table)

%$a$ & distance & \si{\meter} \\
%$P$ & power & \si{\watt} (\si{\joule\per\second}) \\
%Symbol & Name & Unit \\

%\addlinespace % Gap to separate the Roman symbols from the Greek

%$\omega$ & angular frequency & \si{\radian} \\

%\end{symbols}

%----------------------------------------------------------------------------------------
%	DEDICATION
%----------------------------------------------------------------------------------------

%\dedicatory{For/Dedicated to/To my\ldots}

%----------------------------------------------------------------------------------------
%	THESIS CONTENT - CHAPTERS
%----------------------------------------------------------------------------------------

\mainmatter % Begin numeric (1,2,3...) page numbering

\pagestyle{thesis} % Return the page headers back to the "thesis" style

% Include the chapters of the thesis as separate files from the Chapters folder
% Uncomment the lines as you write the chapters

\chapter{Introduction}
This project proposes the use of biosignals, such as heart rate and muscle movement, to augment performances by generating correlated music and lighting.
For instance, a person could perform a movement routine and have the music of that routine be generated in response to their movement,
as opposed to learning a routine based on a preexisting piece of musical composition.
Similarly, lighting could be used to enhance a performance by allowing audiences to have a visual representation of the inner working of a performer's body,
and how that changes based on the state of the performance and the response of the audience.
This opens up room for future exploration into how performance can change when specific movements have additional audio and visual elements,
and how different movements that may not be themselves appealing could correspond to an overall positive performative experience due to these additions.

\section{Background}
Music has been a form of human expression for over 40,000 years~\cite{killin:2018}.
Throughout this time, the creation of music has relied on the skill and dexterity of artists
who have dedicated years to practicing in order to become proficient.
This has presented accessibility challenges for individuals who may be unable to physically perform such actions
or to those who do not have the time required to learn.
This project offers a solution to this challenge by providing a platform for creating music that can be accessible to everyone.
Additionally, this project allows for multifaceted performances due to the lack of physical restrictions on performers
during the dynamic creation of music.

This project is also beneficial to current artists as the addition of lighting control allows for more engaging performances.
Previously, lighting control has been done manually by a skilled lighting technician or automatically triggered by sound.
This project allows the lighting to be controlled directly by the performer, which could allow for much more compelling lighting setups.

A biosignal is a form of communication between biological systems~\cite{semmlow:2018},
they are used in the body to detect various biological events such as muscle contractions and heartbeats~\cite{escabí:2012}.
These signals can be detected using various types of sensors, including electric, mechanical, acoustic, and infrared sensors~\cite{kaniusas:2012}.

\section{Aims}
The aims of this project are to:

\begin{itemize}
        \item Control music and lighting in real-time from biosignals.
        \item Operate effectively in live performance spaces.
        \item Be wearable for an extended period of time without causing physical distress.
\end{itemize}

\section{Literature Review}
\subsection{Introduction}
The use of biosignals to generate music and lighting has been an area of exploration and innovation in the field of live performance technology.
In this literature review, we will examine various approaches and technologies that have been developed in the pursuit of creating music and lighting from biosignals.
We will begin by discussing early attempts at generating music from brainwaves,
and the challenges associated with using electroencephalogram (EEG) signals for live performances.
Then, we will explore the use of other biosignals such as electrocardiogram (ECG), electromyography (EMG), galvanic skin response (GSR), and respiratory rate,
which offer more predictability and discuss why they are better suited to this project.
Finally, we will investigate wireless solutions that enable the integration of biosensors into live performance devices,
and look into existing biosignal based lighting systems, before delving into the Music from Biosignals project;
a project that aims to incorporate biosignals into a wireless platform for live performance.
By reviewing these advancements, we hope to gain insights into the current state of the field and identify areas for further improvement and development.

\subsection{Music from Brainwaves}
The earliest attempt at creating music from brain activity is Alvin Lucier's `Music For Solo Performer'~\cite{Lucier:2010}~\cite{Straebel:2014}.
While this system suffers from various technical issues such as high noise,
the fundamental issue with trying to use electroencephalogram (EEG) to generate any kind of performance signal
is that the output of an EEG is not at all rhythmic and contains a lot of randomness.
Additionally, EEG signals have a high potential for artifacting~\cite{Mannan:2018} and require a large number of electrodes~\cite{Piorecky:2019}.
These factors make EEG signals unsuitable for performance in a live setting and thus they will not be incorporated into the project.

\subsection{Rhythmic Signal Approach}
Other biosignals that could be used are electrocardiogram (ECG)~\cite{Afonso:1999}\cite{Pan:1985},
which is a common and painless measurement that is used to monitor the heart~\cite{Mayo:2023}.
Electromyography (EMG)~\cite{Tanaka:2002}\cite{Young:2013},
which measures electrical activity due to the response of muscles~\cite{Hopkins:2023}.
Galvanic skin response (GSR)~\cite{Kurniawan:2013},
which can show the intensity of emotional changes due to the change in conductance of the skin~\cite{Farnsworth:2018}.
And respiratory rate~\cite{Carlos:2011},
which measures the number of breaths per minute~\cite{Hopkins2:2023}.
These signals have a degree of predictability~\cite{Tahiroğlu:2008} which makes them better suited for use in this project.
There are a number of examples of these kinds of signals being incorporated into live performance settings such as
the `Conductor's Jacket' by Nakra and Picard~\cite{Nakra:1998}, and `Stethophone' by Nerness and Fuloria~\cite{Nerness:2019}.
However, these applications are still limited in their flexibility and use due to the restrictive wired nature of these devices.

\subsection{Wireless Solutions}
Wireless biosensor based performance devices do exist, but there is limited information on them.
Examples of such systems are Yamaha AI's `Transforms a Dancer into a Pianist'~\cite{Yamaha:2018}, and `Emovere' by Jaimovich~\cite{Jaimovich:2016}.
These systems allow performers to elevate their performances by adding an experimental aspect.
However, further exploration could still be done in this space with the addition of lighting control.

\subsection{Biosignal-based Lighting Control}
There is limited activity in controlling lighting using biosignals.
The most relevant research available is Wang's EMG-based Interactive Control Scheme for Stage Lighting\cite{Wang:2022}.
This project uses EMG signals to control lighting in a live performance environment.
However, this project focuses on providing specific control of stage lighting using gestures.
For these gestures to work, the feature extraction algorithm of the device needs to be aware of specific gestures, which are predetermined by the developer.
This limits possible areas of creative exploration, as lighting compositions are predetermined rather than being `found' by the performer.
Thus, there is still room for further developments in this area.

\subsection{The Music from Biosignals Project}
The music from biosignals project has been an ongoing project that attempts to develop and integrate these previously mentioned gaps in literature.
The project has developed an on-body device that acquires biosensors and allows them to be wirelessly transmitted to a PC for processing~\cite{Pierro:2019}\cite{Tran:2022}.
Additionally, software developed in MATLAB has been developed that processes the incoming signals and generates music in real-time~\cite{Chen:2016}\cite{Nicholls:2019}.
Previously, the hardware and software design of the system has been separate and the two parts are yet to be integrated.
This leaves potential future work open in connecting both sides of the project to create one cohesive whole.
Additionally, there has only ever been a musical aspect to the project, allowing further exploration into how lighting could improve the system.

\subsection{Conclusion}
In conclusion, the exploration of generating music from biosignals has seen significant progress in recent years.
While early attempts using brainwaves faced challenges due to their non-rhythmic and unpredictable nature,
other biosignals such as ECG, EMG, GSR, and respiratory rate have shown promise in generating more predictable and rhythmic signals for use in live performance environments.
While a number of projects have incorporated these signals, there is still room in the literature for exploration in applying these signals to lighting systems, in a way that does not limit the creativity of the performer.

\section{Project Plan}
\subsection{Project Objectives}
For this project to be successful, it is necessary to fulfill various requirements.
The requirements for this project are:

\begin{enumerate}
    \item The system must implement lighting control
    \begin{enumerate}
        \item The protocol for lighting control should be available on at least 80\% of performance based lighting fixtures
        \item The lighting control implementation must be compliant to the protocol
    \end{enumerate}
    \item The system output must operate remotely to the acquired sensor data at a range of at least 30 m
    \begin{enumerate}
        \item The acquisition of sensor data must be detached from the system output
        to allow for more complicated off-body setups
        \item The remote system must be compliant with ISO/IEC 15149-1:2014 or equivalent
        \item The wireless system must have a Packet Error Rate (PER) of no more than 1\%
        \item The wireless system must still be functional in a noisy environment
    \end{enumerate}
    \item The system must generate rhytmic music from the acquired biosignals
    \begin{enumerate}
        \item The musical signals need to be correlated to a consistent beat
    \end{enumerate}
\end{enumerate}

\subsection{Assumptions and Constraints}
Assumptions:
\begin{itemize}
    \item The system will be used in an indoor environment with stable temperature and humidity.
    \item The performers will be able to wear biosignal sensors comfortably during their performance.
    \item The sweat from the performers will be mitigated to avoid short-circuiting the electrodes.
    \item The system will exist in a stable performative enviroment.
        \begin{itemize}
                \item The various audio visual elements of the performative enviroment are functional.
                \item The system is being operated by skilled and knowledgable professionals.
        \end{itemize}
\end{itemize}

Constraints:
\begin{itemize}
    \item The total cost of the system cannot exceed \$600.
    \item The size and weight of the whole system must be compact enough to transport in a standard travel bag.
        \begin{itemize}
                \item The system must weigh less than 20kg.
                \item The system must take up less than 50L.
        \end{itemize}
    \item The size and weight of the on-body subsystem must be wearable for several hours without fatigue.
        \begin{itemize}
                \item The on-body element of the system must weigh less than 5kg.
        \end{itemize}
    \item The system must be compatible with a standard lighting protocol.
\end{itemize}

\subsection{Previous Work}
This project is a continuation of the Music from Biosignals project.
As such, a number of these objectives have already been fulfilled, and some hardware and software has been developed.
The extent of these developments are as follows:

\begin{itemize}
        \item Hardware for an on-body device has been developed.
        \begin{itemize}
                \item The hardware has been successfully powered on.
                \item Programmer communication with the microcontrollers has been established.
                \item One of the microcontrollers has been connected to WiFi.
        \end{itemize}
        \item Software for the off-body PC has been written.
        \begin{itemize}
                \item a MATLAB script that generates music using a prerecorded ECG biosignal has been developed.
        \end{itemize}
\end{itemize}

\subsection{Scope}
Given this project outline and the previous project work.
The scope of the project can be defined in~\autoref{tab:scope},~\autoref{tab:scope_stretch}, and~\autoref{tab:scope_out}.
This scope contains stretch goals because some of the previous work had not been completed at the start of the project,
as those students were finishing mid-year.
Therefore, the strech goals exist to allow the necessary project flexibility to account for various outcomes of those student's projects.

\begin{table}[!ht]
    \caption{Project in-scope list}\label{tab:scope}
    \centering
        \begin{tabular}{|l|c|}
        \hline
        \multirow{2}{7em}{Sensors}            & \cellcolor{green!25}Clothing based biosignal acquisition \\ \cline{2-2}
        ~                                     & \cellcolor{green!25}Sensor gain and filtering            \\ \hline \hline
        \multirow{3}{7em}{System Integration} & \cellcolor{green!25}Wireless communication               \\ \cline{2-2}
        ~                                     & \cellcolor{green!25}System tunability                    \\ \cline{2-2}
        ~                                     & \cellcolor{green!25}System testing                       \\ \hline \hline
        MIDI                                  & \cellcolor{green!25}USB MIDI and physical port           \\ \hline \hline
        Lighting Control                      & \cellcolor{green!25}Lighting control over MIDI           \\ \hline
    \end{tabular}

\end{table}

\begin{table}[!ht]
    \caption{Project stretch-goal list}\label{tab:scope_stretch}
    \centering
        \begin{tabular}{|l|c|}
        \hline
        Sensors                               & \cellcolor{orange!25}Wearable sensor design             \\ \hline \hline
        \multirow{3}{7em}{MIDI}               & \cellcolor{orange!25}Compliant MIDI implementation      \\ \cline{2-2}
        ~                                     & \cellcolor{orange!25}MIDI timecode quantisation         \\ \cline{2-2}
        ~                                     & \cellcolor{orange!25}MIDI output timecode               \\ \hline \hline
        \multirow{4}{7em}{System Integration} & \cellcolor{orange!25}Performance application testing    \\ \cline{2-2}
        ~                                     & \cellcolor{orange!25}Bi-directional communication       \\ \cline{2-2}
        ~                                     & \cellcolor{orange!25}Real-time system tunability        \\ \cline{2-2}
        ~                                     & \cellcolor{orange!25}PCB housing design for main board  \\ \hline
    \end{tabular}

\end{table}

\begin{table}[!ht]
    \caption{Project out-of-scope list}\label{tab:scope_out}
    \centering
        \begin{tabular}{|l|c|}
        \hline
        Sensors                               & \cellcolor{red!25}Sensors as independent systems        \\ \hline \hline
        \multirow{2}{7em}{MIDI}               & \cellcolor{red!25}Wireless MIDI                         \\ \cline{2-2}
        ~                                     & \cellcolor{red!25}MIDI threshold points                 \\ \hline \hline
        Lighting Control                      & \cellcolor{red!25}Lighting controller input support     \\ \hline \hline
        \multirow{2}{7em}{System Integration} & \cellcolor{red!25}Cableless system                      \\ \cline{2-2}
        ~                                     & \cellcolor{red!25}User interface for configuration      \\ \hline
    \end{tabular}

\end{table}

\chapter{Lighting Controller}
The first element of the system to be developed once the project began was the lighting controller.
This part of the project was built from scratch, as there was no prior work that had been done in this area.
This made it perfect for the first half of the year,
as other sections of the project required hardware that was currently being developed by the students finishing mid-year.

The lighting controller has a simple function in this system; it maps processed biosignals to lighting position and intensity.
Therefore, it should be able to communicate with the off-body PC and pass various commands through to whatever lights are connect to it.

Different protocols for controlling various lighting fixtures exist.
To determine which should be used for this project, a decision matrix, shown in~\autoref{tab:decision} and~\autoref{tab:decision_cont} was used.
For this project, DMX512 (DMX) was determined to be the most appropriate.

\begin{table}[!ht]
    \caption{Lighting protocol decision matrix}\label{tab:decision}
    \centering
        \begin{tabular}{|l|c|c|c|c|}
        \hline
        ~                    & DMX512 & RDM  & Modbus & 0-10V \\ \hline
        Simplicity    (0.13) & 0.80   & 0.40 & 0.80   & 0.90  \\ \hline
        Expense       (0.14) & 1.00   & 1.00 & 0.70   & 1.00  \\ \hline
        Scalability   (0.10) & 0.90   & 1.00 & 0.90   & 0.20  \\ \hline
        Adoption      (0.16) & 0.90   & 0.80 & 0.50   & 0.30  \\ \hline
        Usability     (0.16) & 1.00   & 0.70 & 0.40   & 0.20  \\ \hline
        Documentation (0.13) & 0.90   & 0.90 & 0.90   & 0.30  \\ \hline
        Licensing     (0.18) & 1.00   & 1.00 & 1.00   & 1.00  \\ \hline
        Total         (1.00) & 0.94   & 0.83 & 0.73   & 0.58  \\ \hline
    \end{tabular}

\end{table}

\begin{table}[!ht]
    \caption{Lighting protocol decision matrix (continued)}\label{tab:decision_cont}
    \centering
        \begin{tabular}{|l|c|c|c|c|}
        \hline
        ~                    & EnOcean & TCP/IP & DALI & BACnet \\ \hline
        Simplicity    (0.13) & 0.40    & 0.20   & 0.30 & 0.30   \\ \hline
        Expense       (0.14) & 0.40    & 0.30   & 0.20 & 0.10   \\ \hline
        Scalability   (0.10) & 0.60    & 0.80   & 0.70 & 0.70   \\ \hline
        Adoption      (0.16) & 0.20    & 0.40   & 0.30 & 0.30   \\ \hline
        Usability     (0.16) & 0.40    & 0.30   & 0.30 & 0.30   \\ \hline
        Documentation (0.13) & 0.80    & 0.80   & 0.10 & 0.10   \\ \hline
        Licensing     (0.18) & 0.00    & 0.00   & 0.00 & 0.00   \\ \hline
        Total         (1.00) & 0.37    & 0.36   & 0.25 & 0.23   \\ \hline
    \end{tabular}

\end{table}

% TODO: ADD PURCHASED VS DIY CONTROLLER DECISION MATRIX
To control DMX fixtures, there are two options, using an off-the-shelf controller or creating a custom controller.
As shown in this decision matrix, the more viable option was to purchase an off-the-shelf controller.

\section{Prototyping Fixture}
To aid in the development of the lighting controller, a prototyping fixture was developed.
This fixture just requires an Arduino and a NeoPixel LED strip to function.
This has the major benefit over a real fixture of being extremely cost effective, as well as being small, and powered over USB.

The prototyping fixture is made up of 8 NeoPixel LEDs with 4 DMX channels each.
The DMX channels are intensity, red channel, green channel, blue channel.
This makes a total of 32 channels (\(4 \times 8\)) for the 8 LEDs.

\subsection{DMX}
The fixture connects to the off-body PC over USB and communicates via serial.
While the device does not strictly require DMX frames in order to function,
DMX was still implemented in software in order to better understand the protocol as well as unify the protoyping fixture with the real controller.


Receives DMX frames using interrupts
and stores them into array for processing.
DMX frames are 512 bytes wide.
A frame consists of the entire DMX `universe' of channels.
Individual channels are never written,
instead the entire `universe' is updated with each frame.
Updates happen continually at a known rate.
This way, fixture are aware if they lose connection to the controller,
since they stop receiving frames.

\subsection{NeoPixels}
LEDs controlled using NeoPixel library~\cite{NeoPixel}.
LEDs can be colored independently using red, green, and blue values.
Each color is represented using a byte.

\subsection{Integration}
Both the DMX channel and color values are represented using bytes.
No additional scaling is required.

Brightness needs to be mapped using
\begin{lstlisting}[language=C]
  (intensity * color) >> 8;
\end{lstlisting}

In this case bit shifting by 8 is equivalent to dividing by 255.
This means that at maximum intensity the result of this calculation is the color value.
While at the minimum intensity the calculation becomes 0.

The values are sent over serial using Base64.
This is so that the carrige return line can be used to mark the end of the incoming values.
This was because the Arduino could become out of sync with what was being sent.
If this happened there is no way of getting back in sync,
because any special character could be interpreted as a regular value.
This is why Base64 is necessary, because it allows for special characters that are seperate
from the designated regular character set.
The values are sent as single decimal digits.
So when they are received they must be decoded back into bytes.

The value of each LED is then encoded across 4 bytes.
So the main program loops, incrementing by 4 each time,
and extracts the discrete bytes, setting the base address of the LED to the derived color.
Once this is done for all the LEDs, they are updated using the show() function.

\section{OpenDMX Controller}
When using expensive hardware in professional settings, it is important to use compliant hardware.
While the prototyping fixture could be used for system verification,
it was decided that existing hardware would be purchased for lighting control.
\textbf{DECISION MATRIX HERE (MIGHT ALREADY HAVE ONE IN MY RESULTS POWER POINT?)}

\subsection{Interface}
The driver provided by the manufacturer was written in C\#.
Existing code for the project is written in MATLAB.
We needed some way of controlling the device from MATLAB to utilise previous code.
\textbf{MORE TO BE WRITTEN ABOUT THIS}

\chapter{On-Body Device}
The on-body device was developed by William Tran in 2021 and consists of two microcontrollers,
a PIC32 and a ESP32, as well as 24-bit ADC for taking biosignal measurements.

At the middle of the year when the project was handed over,
the ESP32 had been programmed to connect to WiFi, but was only programming intermittently,
and the PIC32 had barebones programming on it to enable the ESP32 via the enable pin.

The only other contributions were from Craig Dawson who supplied information and code for programming the PIC32,
as well as attaching wires to specific pins of the PIC32, allowing it to be probed using an oscilloscope.


\section{ESP32}
The ESP32 on the board is the \textbf{ESP MODEL}.
This device is used as a wireless brige between the on-body device and the off-body PC.
It connects to the PIC32 through a Serial Peripheral Interface (SPI) bus.
It is programmed using the Arduino software enviroment.

\subsection{Power}
As stated, the ESP32 was only programming intermittently.
To determine what the cause of this unwanted behaviour was,
the board was tested in the following states in order to measure changes in how the ESP32 programmed.

\begin{itemize}
        \item The board was connected to a lab bench power supply with a non-restrictive current limit.
        \item The boot select switch was held for the extent of the programming cycle.
        \item The boot select switch was held until programming began.
        \item The boot select switch was held from when the device was powered on until programming ended.
        \item The board was powered from a lab bench power supply as well as via USB through an ICD3.
        \item All the same boot select switch options were repeated with the additional power being supplied.
\end{itemize}

From this testing, it was discovered that the ESP32 programs succesfully when it is being adequately powered
and the boot select switch is pressed as the device is being powered on.

The additional power requirement is not due to any external limitations with the power supply,
as the current draw that the supply is measuring is significantly less than the current limit.
Additionally, there is a reduction in current draw from the first supply once the additional power supply is added.
This means that the load is being shared between the two supplies,
as opposed to the first supply being at its max and the second supply provided neccessary additional power.

What his means more broadly is that the problem with programming the device comes from the power distribution of the on-body device.
This can be further verified by measuring the voltage at points on the on-body device.
All of the ICs on the device operate at a 3.3V power level.
When measuring the voltage at the input pins of the ICs and at headers around the board, the voltage appears to be closer to 2.6V.
As we add addtional voltage connections, we can observe the voltage rise.
Although the voltage does not reach 3.3V, the increase in voltage appears to be enough for the ICs to remain powered on.
This appears to be because the power traces on the board are not wide enough.
Because of this, the traces have significant resistance which causes a voltage drop to occur across them once the higher IC currents begin to flow.

For the ESP32, the lower voltage causes a brownout detection feature of the device to activate, causing it to reset.
The effect that has on the device is that it will reset itself out of programming mode, causing the programming to fail.
This is because the ESP32 must be put into programming mode by holding down the boot select switch while the device is initially powered on.
So, even in instances when the device has been succesfully put into this mode,
the device reset caused by the brownout detector reverts it out of this mode.

Once the power supply was supplimented with additional power supplies, this issue became less problematic.
However, an additional issues arose as the ESP32 has to be placed into programming mode as it is powered.
With the addition of these power supplies, there is coordination required in order to get it into this mode.
Since the power is also required to keep the ESP32 from resetting, all of the supplies need to be disconnected and reconnected at the same time,
while the boot select switch is pressed.
Additionally, with this setup there is not way to validate the correct entry into programming mode,
meaning there is not way to know if it has actually been placed into the correct mode until the programming fails.
For prototyping, this becomes extremely cumbersome because this process must be repeated every time there is a change.
As well as whenever the programming fails due to an unexpected reset.

\subsection{Over-The-Air Programming}
\ref{https://citeseerx.ist.psu.edu/document?repid=rep1&type=pdf&doi=5a31727b9b9e5f6a01f5f33a97f787f3db630b32}
When compared to the alternative, Over-The-Air (OTA) programming has a number of benefits.

\begin{itemize}
        \item The device can be programmed regardless of the boot mode.
        \item The device can be programmed without the use of a UART converter.
        \item The device can be programmed without a physical connection to a PC.
\end{itemize}

The downsides to this programming mode is that it adds additional compilation time, as well as runtime time \textbf{?}.
However, the compilation time is already in the range of 40 to 50 seconds, and the addition is less than 5 seconds.
Considering these benefits and drawbacks, it was determined that this programming mode should be implemented.

The first step in implementing this feature was to connect the device to the network.
The ESP32 is programmed using the Arduino IDE, which contains a WiFi library for the ESP32.
The code for simple WiFi device bringout is shown in~\autoref{code:wifi}.

\begin{lstlisting}[language=C++,caption={Arduino code for connected ESP32 to WiFi}\label{code:wifi}]
  #include <WiFi.h>

  void setup() {
    Serial.begin(115200);
    Serial.println('Booting');
    WiFi.mode(WIFI_STA);
    WiFi.begin(SSID, PASS);

    while (WiFi.waitForConnectResult() != WL_CONNECTED) {
      Serial.println(''Connection Failed! Rebooting...'');
      delay(5000);
      ESP.restart();
    }

  }
\end{lstlisting}

Once the device has been connected to the network, OTA programming can be implemented using the ArduinoOTA library.
This consists of including the ArduinoOTA.h header file,
configuring OTA updates using the code shown in~\autoref{code:ota},
and calling ArduinoOTA.handle() in the primary loop of the program.

\begin{lstlisting}[language=C++,caption={Arduino code for configuring OTA updates}\label{code:ota}]
  ArduinoOTA
  .onStart([]() {
    String type;

    if (ArduinoOTA.getCommand() == U_FLASH)
      type = ``sketch'';
    else // U_SPIFFS
      type = ''filesystem'';

  });

\end{lstlisting}

After implementing these libraries, the ESP32 can be programmed using OTA by selecting the relevant device.
One caveat of this is that the OTA updates will only be pushed as long as the OTA handler is called.
So, for instances where the ESP32 is in an infinite loop, or when there is a substantially long delay in the primary loop,
the ESP32 will need to be programmed using the original hardware method.


\section{PIC32}
The PIC32 on the board is the PIC32MX775F512H.
It connects to sensors, the 24-bit ADC, and the ESP32.
It is the main processor on the on-body device and it is programmed in C using MPLAB X v5.00 with an ICD 3.
The ICD 3 must be supplying 3.3V to on-body device in order for the PIC32 to be programmed succesfully.

\subsection{Programming Configuration}
The board designed by Tran \textbf{(REFERENCE HERE)} was designed in 2021.
When the board was being manufactured, there was a global supply chain issue \textbf{(REF?)}.
Because of this, the schematic design of the board was different to what was assembled on the board.
This caused issues when programming the PIC32, because the specific model of PIC had changed.
So, when the programmer connected to the device, it would read a different device ID from what was expected,
and the programming would fail.
Updating the project to use the PIC32MX775F512H solved this programming issue.

The other hurdle in programming the PIC32 was in the clock configuration.
With incorrect clock configuration, the device still programs, but it is not able to be put into debug mode.
Additionally, all intentional delays are based on the clock speed so it must be set correctly for the delays to be correct.

The clock configuration uses PLL~\ref{https://www.ijert.org/phase-locked-loop-a-review}.
It has been configured with a input divider of 10, a multiplier of 16, and an output divider of 8.
With a 16 MHz crystal, the system clock frequency becomes \textbf{CALCULATE}.
\((16 MHz / 10 * 16) / 8 = 50 MHz\)

The system clock frequency was verified by measuring the delay between pulses using an oscilloscope.
The code for delays in the system is shown in~\autoref{code:delay}, and code for generating output pulses is shown in~\autoref{code:pulses}

\begin{lstlisting}[language=C,caption={PIC32 code for adding delays in code execution (in microseconds)}\label{code:delay}]
  void delay_us(unsigned int us) {
    us *= DELAY_CONST;              // DELAY_CONST = SYS_FREQ / 1000000 / 2
    _CP0_SET_COUNT(0);              // Reset Core Timer
    while (us > _CP0_GET_COUNT());  // Wait until Core Timer reaches desired number of clock ticks
  }

  void delay(int ms) {
    delay_us(ms * 1000);
  }
\end{lstlisting}

\begin{lstlisting}[language=C,caption={PIC32 code for measuring system clock speed using delayed pulses}\label{code:pulses}]
  #define TP7 PORTDbits.RD5 // Pin definition for test point 7

  void run() {
    TP7 = 1;
    delay(100);
    TP7 = 0;
    delay(100);
  }
\end{lstlisting}

The measured results for this are shown on the oscilloscope image seen in \textbf{(FIGURE OF SCOPE HERE)}
This figure shows a 100 ms delay between pulses.
This is exactly what the software delay has been set to, which verifies that the system clock frequency has been set correctly,
because the delay relies on the system frequency to calculate the desired amount of clock ticks to wait.

With the system clock frequency set correctly, debugging works on the device and the CPU can be halted and stepped through instructions as the program is running.

\subsection{ESP32 Control}
The ESP32 is controlled by the PIC32 in two ways.

\begin{itemize}
        \item The ESP32 enable pin is connected to pin 14 (RB2) of the PIC32.
        \item The ESP32 HSPI bus is connected to SPI2 of the PIC32
\end{itemize}

The enable pin of the ESP32 just needs to be asserted high, which can be done by the PIC32 using the code shown in~\autoref{code:esp_en}

\begin{lstlisting}[language=C,caption={PIC32 code for enabling the ESP32}\label{code:esp_en}]
  void ESP32_IO_init() {
    TRISBbits.TRISB2 = 0;       // Set ESP32 EN pin as output
    PORTBbits.RB2 = 1;          // Set ESP32 EN pin high
  }
\end{lstlisting}

Without this assertion, the ESP32 is unresponsive to programming and does not perform code execution.

The other form of control is through the respective SPI lines of the two microcontrollers.
The SPI driver for the PIC32 is simple, as that device acts as the `master', which is the typical mode for a microcontroller to operate in.
The PIC32 is configured as a 32-bit master operating in SPI mode 0~\ref{https://www.sciencedirect.com/science/article/abs/pii/B9780123914903000047}
The boardrate generator value has been calculated using the equation \(BRG = (F_{PB} / 2 \times F_{SCK}) - 1\). \textbf{CALCULATE THIS: BRG = 50}.
Finally, the low-level SPI driver is implemented in~\autoref{code:spi_driver}

\begin{lstlisting}[language=C,caption={PIC32 low-level SPI driver}\label{code:spi_driver}]
  uint32_t ESP32_SPI_write(uint32_t data) {
    SPI2BUF = data;                 // Place data we want to send in SPI buffer
    while(!SPI2STATbits.SPITBE);    // Wait until sent status bit is cleared
    uint32_t read = SPI2BUF;        // Read data from buffer to clear it

    delay_us(5000);                 // Required delay for data transmission
    return read;
  }
\end{lstlisting}

The function of the code is relatively straightforward.
The memory location of the SPI2BUF variable is mapped to the SPI 2 peripheral.
When written to, the data at this address is written into a transmit buffer that queues the data for SPI transmission.
When read from, data that has been received by the SPI peripheral is taken from the receive buffer.
Between these two operations the processor waits for the SPITBE status flag to be set.
This flag corresponds to the transmission buffer being empty, and is set once transmission has been completed and data has been received.

Additionally, a delay is required between SPI writes in order to stop data from becomming corrupt.
This delay was embedded into the low-level driver to make eventual performance optimisation centralized,
since this delay is a clear cost to performance and by far the cause of the most communication slowdown.
This delay is most likely necessary due to the operation of the SPI chip select line.
Specifically, the way the chip select line does not reset between SPI transmissions without adequate delays between writes.
This could be solved by manually asserting the chip select line instead of allowing the peripheral to control it.
However, due to time restrictions it was decided that features should be prioritized over performance,
and thus the driver was implemented using significantly slower delay.

The ESP32 SPI configuration was less straightforward.
As microcontrollers are typically the devices that coordinate communication between various `dumb' sensors,
they almost always act as the SPI master.
However, since the on-body device uses multiple microcontrollers communicating via SPI,
one of these devices needs to act as a slave.
Since the PIC32 is what communicates to the sensors, and the ESP32 only acts as a wireless transmitter,
the PIC32 was configured as the SPI master, leaving the ESP32 as an SPI slave.
There no official support for this in the native development enviroment, so either a custom or third party library must be used.
In the interest of getting as much of the project functional as possible, it was decided that a third party library would be used \textbf{DECISION MATRIX?}.
However, there are potential peformance improvements that could be achieved if a custom library was used, such as reducing the PIC transmission delays,
so it is recommended that this option is looked into in the future.

The library that was used is the ESP32DMASPI~\ref{https://github.com/hideakitai/ESP32DMASPI} library.
This library is based on the official driver from the manufacture~\ref{https://docs.espressif.com/projects/esp-idf/en/latest/esp32/api-reference/peripherals/spi_slave.html#spi-slave-driver}.

How do I describe my workarounds in order to get this to work aside from just saying that that's what I did?
SPI implemented using task based DMA receiving.
Tasks run in seperate threads, which frees up main thread to only process OTA updates.
One task runs continously waiting for SPI transmission~\autoref{code:esp_spi}.

\begin{lstlisting}[language=C++,caption={ESP32 code for receiving SPI data}\label{code:esp_spi}]
  void task_wait_spi(void* pvParameters) {
    while (1) {
      ulTaskNotifyTake(pdTRUE, portMAX_DELAY);
      slave.wait(buffer, BUFFER_LENGTH);
      xTaskNotifyGive(task_handle_process_buffer);
    }
  }
\end{lstlisting}

Once data has been received, a different task handles processing the incomming data~\autoref{code:esp_spi_processing}.

\begin{lstlisting}[language=C++,caption={ESP32 code for processing SPI data}\label{code:esp_spi_processing}]
  void task_process_buffer(void* pvParameters) {
    while (1) {
      ulTaskNotifyTake(pdTRUE, portMAX_DELAY);
      print_array(buffer, slave.available());
      slave.pop();
      xTaskNotifyGive(task_handle_wait_spi);
    }
  }
\end{lstlisting}

\textbf{DESCRIBE MORE DETAIL ABOUT HOW THESE WORK.}

This should have been the end of the SPI implementation,
however the SPI communication did not work with just this code.
To get the communication working some additional functions were required on the PIC~\autoref{code:spi_additional_functions}.

\begin{lstlisting}[language=C++, caption={PIC32 additional SPI functions}\label{code:spi_additional_functions}]
  void ESP32_SPI_write_4byte(uint8_t b1, uint8_t b2, uint8_t b3, uint8_t b4) {
    uint32_t word = ((uint32_t)b1 << 24)
                  | ((uint32_t)b2 << 16)
                  | ((uint32_t)b3 << 8)
                  | (uint32_t)b4;

    ESP32_SPI_write(word);
  }

  void ESP32_SPI_write_byte(uint8_t data) {
    ESP32_SPI_write_4byte(data, 0, 0, 0);
  }
\end{lstlisting}

Then, the only way to send data was to write a single byte at a time.
What is interesting about this is that the single byte is located at the front of the 4 byte word.
This implies that the ESP32 was not able to receive 32-bits, and instead was just receiving the first 8-bit word.
However, if the SPI peripheral of the PIC32 was put into 8-bit mode (and all respective code changed to fit that mode),
the ESP32 would receive nothing.
So, the ESP32 required the PIC32 to send 32 SPI clock pulses but would only receive the first 8 data bits.
This same behaviour was also present when using a task based approach or a polling approach and when DMA was or was not in use.
It is a substantial issue because it adds a 75\% overhead to the communication system.
This, and the required delay in the low-level driver, are the primary candidates for future optimisations.

However, despite these performance issues, the drivers function together.
Allowing abritary bytes to be shared between the two devices.


\section{ADS1294R}
The ADS1294R is a 4 channel, 24-bit, delta-sigma analog-to-digital converter with additional features to support electrocardiogram and electroencephalogram measurements.
This device is the primary sensor front end.
It communicates with the PIC32 via SPI as well as several hardware control lines.

\subsection{Design Differences}
Like the PIC32, the schematic for the ADS1294R were slightly different to what was assembled due to chip shortages.
The schematic showed the devices as an ADS1298R, which is the 8 channel version of the device.
This is significant because the device requires a specific number of SPI clock pulses to be sent that correlates to the number of channels on the device.
So sending the wrong number of clock pulses will generate undesired results.
Additionally, the device ID is different which can cause some confusion when intially trying to configure the device.

\subsection{SPI}


Also appears to be additional connection from nDRDY to DGND with some capacitors.

Not 100\% sure if this is actually happening though because only image of PCB design
low res screenshot of only top side of board. Very hard to follow traces.
As far as I can tell it seems reasonable that each trace routed from the PIC to the ADS
is correct. Not really possible to check as it is a ZXG package (basically a BGA package).

Also pin 43 of the PIC is connected to DOUT which according to the datasheet of the PIC
is actually the SPI chip select pin not the data input pin.

Might have to write a custom driver that just pulls whatever pin the actual chip select
is connected to low, writes whatever the equivalent SCK pin is high and write/reads
the corresponding SDO and SDI pin manually.

Not sure if any of this is actually managable anyway because the reset switch
is a push button and doesn't appear to have any way of switching it via the PIC.

Can't even really tell where the reset pin goes because it is grounded to the sleve of
the 3.5mm jack?!? Whatever is routed to the switch is on the underside of the board
that I don't have access to.

Currently trying to change all the configurations to see if any combinations work better.
If this doesn't work I can look at the actual altium schematics I got from Kenneth.
Thinking first I'll just try and get the actual ID from the chip
since it is easy to verify.
Going to play around with the driver, add delays between reads and writes and try different
SPI configs to see if anything makes it respond.
Also will try same thing after physically pressing reset switch because maybe it needs
hardware reset.

My scope probe wasn't working because of the x10 setting on the probe was x1.
I actually have managed to get something out of the ID register.

It should be    11010000
It actually is  01100000

This was gotten by setting SPI bits SMP = 0, CKE = 1, CKP = 0, and pressing
the reset button for the chip while using the scope to send a signal at the moment it
would have to be reset in software.
Managed to repeat this at least one more time. Attempting without pressing hardware reset.
Still works without hardware switch being pressed. Not correct but better than nothing.
Only thing that I changed was using the scope to correctly set the delay constant.

Used scope to measure space between pulses then set a specific pulse delay.
I think previously the delay number was overflowing potentially because it was so large.
Potentially the delay was way too short and the timing wasn't correct.

Can get 11000000 if I set SMP = 1
Still missing bit 4 which should be 1 no matter what.
Trying without hardware reset. Still works.

Setting CKP causes ID to be 0, so that should 100\% be reset.

Resetting it and running again causes ID to be 0
Pressing reset while it's running fixes the issue.
Still not what it 100\% should be but it's close.

Increasing the delay amount did not change anything.
Seems like hardware reset can be pressed midrun and it recovers nicely

Without delays it often does not read correctly.
Going to add a single delay in the SPI write function to set max speed.

Going to reduce the delays in the init function. Everything seems to work fine.
Going to try and remove them completely. Seems to work correctly still.
Appears that the singular delay in the SPI write function was enough and nothing else
needs it now. Probably good to do this with the ESP SPI write so that the delays are
only in a single place and can be easily optimised.

Added a substantial amount of extra delay to both to guarantee consistency, can easily
be optimised later.

Changing SPI3BRG as well to see if that makes any kind of difference.
Seems like it can be about any value and it still works the same.
Calculations say it should be greater than 2 so I'll set it to 4 for safety.
However, still works without propogating errors with a value of 1 which
should be too fast.

Wondering if the GPIO init I am doing has an effect. Turning it off causes ID to be 0.
Which is good because it means that I am actually doing something.

Going to go through init function and see what turning various things on/off
causes the system to do.
`CLK\_SEL = 0' causes the chip to stop responding.
Not writing SDATAC doesn't appear to change anything but that could be because I'm not
doing anything that needs registers set.
`START\_PIN = 0' causes chip to stop responding.

Ok now that I have a baseline that I can consistenly get to I'm going to try and do more.


\subsection{Reading Data}
When reading data it doesn't seem like the DRDY pin ever actually goes low.
Even when not sending read data continuous it is always high.
Not really sure if I can trust this though because the pin isn't exposed
So I'm setting a test point high if it is read high by the microcontroller and it may
just be staying high because by the time it can poll again it has already read all the data
and thus the chip is ready to put more data out.

Going to see what happens when I try and read without setting RDATAC.
Ok so issue now is that I am always reading 11000000 even when not reading the ID.

This is really strange to me because it should be shorted so reading 0.
Wonder if that also means that the ID I was reading is not actually correct since it wasn't
exactly what it should have been anyway and it matches what the chip sends continuously.

From my calculations I am reading 15 bytes which is 120 bits which I belive is correct.
If I don't short the input it is still 11000000

Ok so something definitly not right because even once read a few bytes the `DRDY'
pin is still high when it should be low.

It's strange because the DRDY pin is active low, in software I am checking `DRDY\_PIN == 0'
which made me think that maybe the chip just isn't on at all hence why I'm always seeing
`data\_ready()' as true.
But this isn't the case since it becomes false if I hold down the reset pin.
Which means the physical pin is being pulled high when the chip is put into reset which
seems super strange to me because I would assume everything would just go to 0.
At least in this case I can confirm that it is actually working.

It seems wrong that sending the stop data read continuous command doesn't stop the chip
from setting it's data ready pin. Not super logical to me because shouldn't that only be
true when the data is continuously being streamed.

I think the fact I am getting 192 (AKA 11000000) all the time, probably means that all the
register writes I am doing are not actually correct. What I should do I think is go back
to the ID register and attempt to get that to send correctly.

I am not entirely sure where I should start because it seems like this should be working.
I'm going to go back to the SPI config and see what I can maybe do.

Also might be worth looking at software libaries for Arduino to see what they do in say
a simple ID test script.
Maybe there is something here I am missing that is really obvious.

I think it is the power-on timing. Most likely due to hte lack of /RESET pulse since
that was never connected to the PIC.

Considering that, I think it's fair to assume that the timings may still be too short.
Might be worth looking at existing Arduino library for how they setup their SPI.
Also worth looking at schematics to verify there weren't any revision changes with how
the chip RESET is connected (although I doubt there would be).

Looking at presumably functional Arduino library:
clock polarity 0, clock phase 1, output edge rising, data capture falling

What this actually means:
Idles on logic low, data transmission from clock idle to clock active,
data shifted out on rising clock edge, data sampled on falling clock edge

\textbf{ADDITIONALLY}: Data is MSB first with a datarate of 4,000,000 (AKA 4MHz)

Also just realised it doesn't actually matter what I set the baudrate generator to because
the system clock is 5Mhz so it will literally always be able to keep up.


\subsection{Inconsistencies}
Interestingly enough writing START and STOP appears to work.
So the actual writing to SPI seems to be correct.

Very weird because it seems like there are two versions of the design.
The one that got made appears to be older and worse. So not really sure what's going on.
Perhaps this is the beginning of a revision 2?
But I also cannot find the altium files for it, only a single pdf remains.
Shame because the newer one has a bunch of LEDs on it that would make this way easier.

So what we know is that write must work. As writing START and STOP commands have
an effect.
What is currently unknown is if the reading and writing register commands work correctly.
I really don't see how they wouldn't because I've hand checked the bits during debugging.
Really the only thing I can think at this stage is that maybe the PIC is sending in 32-bit
mode. That doesn't really make sense since it's configured to 8-bit mode but maybe it
still sends extra or something (this would be very silly).

I cannot think of anything else to check short of soldering to the pins and checking
the physical signals that are being send both ways. I'm sure if I did that I would be able
to verify it almost immediately... Should definitly bring bodge wire on Tuesday and make
use of the good soldering irons, flux, microscopes, and nice oscilloscopes at uni.

Other than that it could also be the power on cycle needs specific lines held low/high
and I just do not have digital access to those pins.

Or worst case the entire board has been designed so badly that the chip is having weird
power issues. Looking at the altium board files it definitly does not look great.
At least there is a lot to include in ADC challenges for my results.


The status word that I am expecting contains the following
`1100 LOFF\_STATP[7:0] LOFF\_STATN[7:0] GPIO[7:4]'
So maybe the received 11000000 could be the beginning of that status word but then I
am not reading any more or something?

GPIO register looks like `DATA[4:1] CONTROL[4:1]' with 0 being outputs and 1 being inputs
for the control section

So I am not able to write the register. I think at this stage what must be wrong is my
register writing/reading.
I think the actual write commands are working
because if I write sleep, wakeup, start, stop, or reset
the chip has the response I would expect it to have.
I am not able to verify SDATAC or RDATAC because the data I am then reading is wrong.
It also doesn't seem to stop it from giving me data if I send either of them.
Maybe I should try RDATA and try to just do a simple read see if that makes any difference.

Otherwise, I think the write byte works so I don't see why writing the read/write register
command then the address of the register is not working.

It could also be that the SDATAC, RDATAC, and RDATA commands do not work for some reason
This would mean that all my attempts to read and write registers is futile because
the chip is not going to respond to those commands as it boots into RDATAC mode.
But even so the chip does not give any readings, just 192 (AKA 11000000) continously
so I have no idea what that is supposed to mean.


\subsection{Solution}
Ok so the issue actually came from the CS line. The line was being driven active for each
command but because the ADC had multi-byte commands it was expecting the CS line to stay
active for the entire duration.
What was instead happening was the CS line was going inactive
and causing the ADC to reset mid command which meant it was only responding to single byte
command hence why RESET, STANDBY, WAKEUP, etc were working correctly.

Issue now is my driver has an issue because I'm reading all 0s for the
ID but on the scope I can see the response is correct.

Issue was that the SPI buffer holds the previous transmitted value so need to clear it once
before reading again to get actual value.

Also need to make sure to use special `write\_cmd()' function as it contains the neccessary
chip select setting/resetting and it will not respond to commands otherwise.

\chapter{Integration}
\section{Methods}
The integration of the system was performed by individually implementing each sub-section
and ensuring that the sub-section's outputs are compatible with adjacent sub-section inputs.

With this methodology in mind, the system was integrated in the order of outputs to inputs.
This is so that with the edition of each new sub-section, the system response to the new sub-section can be verified.

First, the lighting controller was implemented, as shown in the Lighting Controller chapter.
Then, software for the off-body PC was written that could communicate with the lighting controller,
and control the prototyping fixture based on a prerecorded dataset.
The dataset was that of an electrocardiogram sampled at 360 Hz.
The lights were controlled by pulsing their intensity at the peaks of the electrocardiogram.
Then, the software was modified so that the data being processed could be received as a continuous stream, instead of a fixed dataset.

Communication between the on-body device and the off-body PC has already been documented in the On-Body Device chapter.
This communication was expanded upon by receiving the data into the same continuous stream that had already been established with the dataset,
thus allowing the on-body device to control the prototyping fixture through peaks in the transmitted data.

The on-body device needs to collect different types of data with varying sample and transmission rates.
Much like implementing the prototyping fixture for the lighting controller where the end of each DMX frame needed to be marked,
the varying data sizes were implemented using special characters to mark the end of data transmission using base64.
Additionally, packet headers were created to specify the different kinds of data that are being sent
(implementation was prior to the board power distribution issues).


\section{Results}
The off-body PC has a fixed length first-in-first-out buffer that can be filled with either testing data or streamed data from the on-body device.
The received data can control the prototyping fixture, with various methods of control through data processing.
Beat detection was implemented on the testing dataset,
allowing the prototyping fixture to increase intensity at each peak of the input data.

\begin{figure}[!ht]
  \caption{Fixed length FIFO data buffer with testing dataset}\label{fig:matlab_packet_test2}
  \centering
  \includegraphics[width=1\columnwidth]{chapters/development/MATLAB/FULL_FILLED}
\end{figure}

The implementation of varying data sizes and the inclusion of packet headers allows data to be effectively transmitted and received.
An unintended benefit of this is that a debug packet header can be specified, and since the data can be variable length,
a fully functional `printf' function on the PIC32 can be used to send arbitrary debugging messages to the off-body PC.

As was presented in the On-Body Device chapter, the PIC32 and ESP32 are able to communicate with the off-body PC at a speed of 1.45 kbits/s.
At a sampling rate of 50 Hz~\cite{Ajdaraga:2017}, the bandwidth is high enough to send 29 bits.
For instance, you could send 8-bits of header information and 16-bits of electrocardiogram data, while maintaining sampling and transmitting at 50 Hz.


\section{Discussion}
To reduce the amount of hardware that is required to be worn by the performer,
the bulk of the processing is to be done on an off-body PC.
This PC communicates wirelessly with the on-body device,
receiving sensor data from the various biosignals that the device is measuring.
The PC then must process the data and coordinate the music and lighting generation.

% TODO: WOULD BE GOOD TO HAVE A DECISION MATRIX HERE
MATLAB was used as the off-body PC language for processing because of ease of use, versatility,
and previous usage on this project.

Since MATLAB functions operate on arrays and matrices,
the desired behaviour of our program is to store a length of data and process it all together,
rather than try to process each sample individually as it arrives.
To keep the system responsive, the buffer is kept at a fixed length,
so that over time the processing does not incrementally take longer due to the increase in data to process.
To achieve this, the number of samples added to the front of the buffer need to be removed from the rear of the buffer each time a new sample is received.
An `offline' version of this program can be made without the need for full system integration using previously saved sampled data.
Since this data is the same as what we will be eventually expecting to see,
The processing code that we write for this test data will also function similarly on the real data.

For testing, an electrocardiogram (ECG) data-set from a previous iteration of the project was used.
The load command will load a number of different ECG data-sets with that all have different beats per minute (BPM),
and the desired ECG signal for testing can be selected by changing the the assignment of the ECG variable.
For example, to load a 60 BPM ECG signal, the assignment could be changed to ECG = ecgdata360Hz\_hrmean60.
The specific data-sets that are available can be seen in the workspace tab of MATLAB once the load command has been executed.
Additionally, a loop rate is calculated to simulate the sensor sampling rate.
Since the testing data-set sample rate is known, it can be easily calculated from that.

Once the data-set has been imported, the program can loop through and generate example packets for testing.

The first thing that is done is the constants and variables definitions.
In this context, packet refers to the small subset of the ECG signal that is going to be added to the larger fixed length array.
That fixed length array is labelled `data' in this example.

The packet index is the data position from the ECG data-set where data will be sampled from.
Packet length determines how many samples at a time are shifted in and out of the data array,
this is to simulate the device transmitting multiple samples at a time.

When the packet index reaches the end of the ECG data-set, the index resets back to the beginning.
This is so the test program can run continuously, regardless of how big the test data-set actually is.
When the index resets, there is a break in continuity of the signal. However, this is a known issue and can be easily detected and ignored.

% TODO: FIGURE SHOWING DATA ENTERING AND EXITING FIXED LENGTH ARRAY
The fixed data array can then be updated by first shifting the current data to the left, disposing of packet length worth of data at the beginning.
Then, the same amount of data, from the packet, can be appended to the end of the array.
Plots of this data are shown in~\autoref{fig:matlab_packet_test1} and\autoref{fig:matlab_packet_test2}.

With this setup, the off-body PC is ready to process an incoming signal as a fixed sized array of data.

% TODO: WOULD BE GOOD TO HAVE SOME RESULT FROM THIS IN SOME CAPACITY
This code calculates the BPM of the signal by measuring the gaps between the peaks of the ECG QRS signals.
It can be verified by comparing the measured BPM value with the given value from the data-set.
When the system is integrated, as long as the data length matches and the sensor data is accurate,
this same code will produce similar results.

With all the block of the system functional,
the first step of system integration is to establish communication between the off-body PC and on-body device.

On the on-body device, the ESP32 acts as the wireless bridge.
As this device requires a connection to WiFi for OTA updates, the wireless protocol that was chosen to be implemented was TCP/IP.
This is because the device already has a TCP/IP requirement and the only other choice for wireless communication was Bluetooth, which does not have the range or speed that TCP/IP has.

In order to get these devices to communicate, inbound and outbound rules need to be defined in the off-body PC firewall.
This is to allow connections on the specified port, since most PC ports are typically closed for security reasons.
The port settings can be seen in~\autoref{fig:firewall}.

\begin{figure}[!ht]
  \caption{TCP/IP firewall port settings}\label{fig:firewall}
  \centering
  \includegraphics[width=1\columnwidth/2]{chapters/development/FIREWALL}
\end{figure}

With data being sent between the ESP32 and the off-body PC, the next device to connect is the PIC32 to the off-body PC.
Since the SPI communication between the PIC and the ESP has already been established,
the process buffer task can be updated to pass-through the received data to the off-body PC.

Sending individual bytes between devices is now possible.
Currently, this only works as long as the value fits into a predetermined number of bytes.
Additionally, once the system is running with non-known values, there is no easy way to keep the data synchronized.
For instance, if a byte of data was lost, the received data would be in the incorrect place
and there would be no way of knowing.
This could be solved by sending a known byte sequence at the beginning of transmission.
However, there would be no differentiation between that byte sequence and any arbitrary data that may be being sent.
Another approach is to limit the character set so that specific byte values can correspond to `special' characters.
For example, the decimal value of 10, which corresponds to the newline character, could be used to mark the end of transmission.
That way, even if the transmitter and receiver become out of sync, the system has a way of recovering.
There are a number of character sets that have already been developed.
For this project, base64 was implemented due to its lower overhead when compared to ASCII, its human readability to aid debugging, and its good documentation.

This code takes the inputted data and limits it to only use characters from the encoding table.
After the data has been encoded, it is sent to the ESP32 for writing to the off-body PC.

With this implemented, data that is sent to the off-body PC is consistently received correctly and the devices do not get out of sync.
This encoding function allows for arbitrary data to be sent between the PIC32 and the off-body PC.
One major benefit of this is that it allows custom wrapping of a printf function, along with the previously implemented packet writing function,
to give the user a custom debugging function that allows any number or string to be formatted and sent to the off-body PC.

Then, the program can be debugged using printf.
This allows the device to continue running while providing feedback, rather than needing to be put into debugging mode and stepped through manually.
The primary benefit of this is that it provides information without disrupting timings such as the use of breakpoints does, also this allows debugging in callback functions,
which the debugger conventionally cannot step into.

The final part of the project is the testing and validation.
While all of the independent parts of the system had been tested in a vacuum,
unfortunately when the system was fully operational, the device failed due to power supply issues.
The issues are related to what was seen earlier with the ESP32 power issues.
As more devices came online, the current draw across on-body device's power plane increased, which subsequently increased the voltage drop.
Eventually, this voltage drop became too large and the devices started to lose power.
Thus, the testing and validation of the system remains unfinished, as the device was no longer responsive at the point when testing began.

\chapter{Future Work}
There are a number of recommendations for future work on the project.

Firstly, the on-body device should be revised.
The device that has previously been developed is a good starting point.
However, the next revision should include a more prototyping friendly design (more LEDs, test points, etc.),
improved power distribution, and potentially simplified wireless communication,
would allow the project to be developed further.

Another aspect of future work that could be undertaken is the development of wearable sensors.
There is room for a lot of research and experimentation when it comes to ergonomic sensor design.

Further developments can be made in the software aspect of the project.
Generally, the communication speeds between all the devices could be increased.
At present, speeds are lower than the maximum capacity of the devices to ensure communication is consistent.
There are also a lot of unnecessary overheads on a number of the communication lines.
This was to get the devices connected, but further work could be done to optimize this communication.

Lastly, the system could be tested in a real live performance space.
There is obviously work that would need to be completed before that would be possible.
However, with that work complete, a live performance test with feedback from the performers would be of immense value.

\chapter{Conclusion}
This project was a continuation of the Music from Biosignals project.
It focused on developing the hardware for biosignal acquisition, and a controller for controlling lighting fixtures.
The project was successful in creating the lighting controller, as well as programming the hardware.
However, due to previous board design issues, the devices were not able to be powered together.
Therefore, while the device to device connections on the board were tested and working,
the fully system integration was not able to be completed.

This project succeeded in demonstrating the design flaws of the on-body device, while also providing a number of recommendations for a future redesign.


%----------------------------------------------------------------------------------------
%	BIBLIOGRAPHY
%----------------------------------------------------------------------------------------

\printbibliography{}

%---------------------------------------------------------------------------------------
%	THESIS CONTENT - APPENDICES
%----------------------------------------------------------------------------------------

\appendix % Cue to tell LaTeX that the following "chapters" are Appendices

% Include the appendices of the thesis as separate files from the Appendices folder
% Uncomment the lines as you write the Appendices

% TODO: eventually need to get this working
% for some reason makefile does not allow access to this directory

\chapter{Gantt Chart}\label{appendix:gantt}

\includegraphics[width=1\columnwidth]{chapters/project_plan/figures/Gantt_Chart}

\chapter{Prototyping Fixture Code}\label{appendix:prototyping_fixture}

\begin{lstlisting}[language=C++]
#include <stdint.h>
#include <stdio.h>
#include <string.h>
#include <Adafruit_NeoPixel.h>
#include <Base64.h>

#define PIXELS_PIN 6
#define NUM_PIXELS 8
#define ADDR_PER_PIXEL 4
#define DMX_LENGTH (NUM_PIXELS * ADDR_PER_PIXEL)
#define UNIVERSE_SIZE 32

#define INPUT_BUFFER_LENGTH 1024
#define BASE64_LENGTH 128
#define RAW_LENGTH 95

Adafruit_NeoPixel pixels(NUM_PIXELS, PIXELS_PIN, NEO_GRB + NEO_KHZ800);
uint8_t dmx_values[UNIVERSE_SIZE];

char input_buffer[INPUT_BUFFER_LENGTH];
char decode_buffer[BASE64_LENGTH];
uint16_t input_index = 0;
uint8_t new_data = 0;

void setup() {
  Serial.begin(115200);
  pixels.begin();
  pixels.clear();
  pixels.show();
}

void serialEvent() {
  while (Serial.available()) {
    if (input_index > INPUT_BUFFER_LENGTH) { input_index = 0; } // Should never hit this
    char c = Serial.read();

    if (c == '\n') {
      memset(decode_buffer, '\0', BASE64_LENGTH);
      memcpy(decode_buffer, input_buffer, input_index);
      new_data = 1;
      input_index = 0;
      continue;
    }

    input_buffer[input_index++] = c;
  }
}

void loop() {
  if (new_data) {
    new_data = 0;

    int decoded_length = Base64.decodedLength(decode_buffer, BASE64_LENGTH);
    char decoded_string[decoded_length];
    Base64.decode(decoded_string, decode_buffer, BASE64_LENGTH);

    int array_index = 0;
    for (int i = 0; i < decoded_length; i += 3) {
      char val[3];
      val[0] = (decoded_string[i+0]);
      val[1] = (decoded_string[i+1]);
      val[2] = (decoded_string[i+2]);

      dmx_values[array_index++] = atoi(val);
    }

    for (int addr = 0; addr < DMX_LENGTH; addr += ADDR_PER_PIXEL) {
      uint8_t i = dmx_values[addr + 0];
      uint8_t r = dmx_values[addr + 1];
      uint8_t g = dmx_values[addr + 2];
      uint8_t b = dmx_values[addr + 3];

      uint32_t color = pixels.Color(
        (r * i) >> 8,
        (g * i) >> 8,
        (b * i) >> 8
      );

      pixels.setPixelColor(addr / ADDR_PER_PIXEL, color);
    }

    pixels.show();

  }
}

\end{lstlisting}

\chapter{DMX Server Code}\label{appendix:dmx_server}

\begin{lstlisting}[language=C++]
using System;
using System.Runtime.InteropServices;
using System.IO;
using System.Threading;

namespace DMXServer
{

    public class OpenDMX

    {

        public static byte[] buffer = new byte[513];
        public static uint handle;
        public static bool done = false;
        public static int bytesWritten = 0;
        public static FT_STATUS status;
        public static Thread thread;

        public const byte BITS_8 = 8;
        public const byte STOP_BITS_2 = 2;
        public const byte PARITY_NONE = 0;
        public const UInt16 FLOW_NONE = 0;
        public const byte PURGE_RX = 1;
        public const byte PURGE_TX = 2;



        [DllImport("FTD2XX.dll")]
        public static extern FT_STATUS FT_Open(UInt32 uiPort, ref uint ftHandle);
        [DllImport("FTD2XX.dll")]
        public static extern FT_STATUS FT_Close(uint ftHandle);
        [DllImport("FTD2XX.dll")]
        public static extern FT_STATUS FT_Read(uint ftHandle, IntPtr lpBuffer, UInt32 dwBytesToRead, ref UInt32 lpdwBytesReturned);
        [DllImport("FTD2XX.dll")]
        public static extern FT_STATUS FT_Write(uint ftHandle, IntPtr lpBuffer, UInt32 dwBytesToRead, ref UInt32 lpdwBytesWritten);
        [DllImport("FTD2XX.dll")]
        public static extern FT_STATUS FT_SetDataCharacteristics(uint ftHandle, byte uWordLength, byte uStopBits, byte uParity);
        [DllImport("FTD2XX.dll")]
        public static extern FT_STATUS FT_SetFlowControl(uint ftHandle, char usFlowControl, byte uXon, byte uXoff);
        [DllImport("FTD2XX.dll")]
        public static extern FT_STATUS FT_GetModemStatus(uint ftHandle, ref UInt32 lpdwModemStatus);
        [DllImport("FTD2XX.dll")]
        public static extern FT_STATUS FT_Purge(uint ftHandle, UInt32 dwMask);
        [DllImport("FTD2XX.dll")]
        public static extern FT_STATUS FT_ClrRts(uint ftHandle);
        [DllImport("FTD2XX.dll")]
        public static extern FT_STATUS FT_SetBreakOn(uint ftHandle);
        [DllImport("FTD2XX.dll")]
        public static extern FT_STATUS FT_SetBreakOff(uint ftHandle);
        [DllImport("FTD2XX.dll")]
        public static extern FT_STATUS FT_GetStatus(uint ftHandle, ref UInt32 lpdwAmountInRxQueue, ref UInt32 lpdwAmountInTxQueue, ref UInt32 lpdwEventStatus);
        [DllImport("FTD2XX.dll")]
        public static extern FT_STATUS FT_ResetDevice(uint ftHandle);
        [DllImport("FTD2XX.dll")]
        public static extern FT_STATUS FT_SetDivisor(uint ftHandle, char usDivisor);


        public static void start()
        {
            handle = 0;
            status = FT_Open(0, ref handle);
            thread = new Thread(new ThreadStart(writeData));
            thread.Start();
            setDmxValue(0, 0);  //Set DMX Start Code
        }

        public static void stop()
        {
            if (thread != null && thread.ThreadState != ThreadState.Stopped)
            {
                thread.Abort();
            }

            status = FT_Close(handle);
        }

        public static void setDmxValue(int channel, byte value)
        {
            if (buffer != null)
            {
                buffer[channel + 1] = value;
            }
        }

        public static void writeData()
        {
            while (!done)
            {
                initOpenDMX();
                FT_SetBreakOn(handle);
                FT_SetBreakOff(handle);
                bytesWritten = write(handle, buffer, buffer.Length);
                Thread.Sleep(20);
            }

        }

        public static int write(uint handle, byte[] data, int length)
        {
            IntPtr ptr = Marshal.AllocHGlobal((int)length);
            Marshal.Copy(data, 0, ptr, (int)length);
            uint bytesWritten = 0;
            status = FT_Write(handle, ptr, (uint)length, ref bytesWritten);
            return (int)bytesWritten;
        }

        public static void initOpenDMX()
        {
            status = FT_ResetDevice(handle);
            status = FT_SetDivisor(handle, (char)12);  // set baud rate
            status = FT_SetDataCharacteristics(handle, BITS_8, STOP_BITS_2, PARITY_NONE);
            status = FT_SetFlowControl(handle, (char)FLOW_NONE, 0, 0);
            status = FT_ClrRts(handle);
            status = FT_Purge(handle, PURGE_TX);
            status = FT_Purge(handle, PURGE_RX);
        }

    }

    /// <summary>
    /// Enumaration containing the varios return status for the DLL functions.
    /// </summary>
    public enum FT_STATUS
    {
        FT_OK = 0,
        FT_INVALID_HANDLE,
        FT_DEVICE_NOT_FOUND,
        FT_DEVICE_NOT_OPENED,
        FT_IO_ERROR,
        FT_INSUFFICIENT_RESOURCES,
        FT_INVALID_PARAMETER,
        FT_INVALID_BAUD_RATE,
        FT_DEVICE_NOT_OPENED_FOR_ERASE,
        FT_DEVICE_NOT_OPENED_FOR_WRITE,
        FT_FAILED_TO_WRITE_DEVICE,
        FT_EEPROM_READ_FAILED,
        FT_EEPROM_WRITE_FAILED,
        FT_EEPROM_ERASE_FAILED,
        FT_EEPROM_NOT_PRESENT,
        FT_EEPROM_NOT_PROGRAMMED,
        FT_INVALID_ARGS,
        FT_OTHER_ERROR
    };

}

\end{lstlisting}

\begin{lstlisting}[language=C++]
using System;
using System.Threading;
using System.IO.Ports;
using System.Linq;

namespace DMXServer
{
    public class ArduinoDMX
    {
        static SerialPort port;
        static Thread thread;

        static byte[] buffer = new byte[8*4];

        public void start()
        {
            if (port != null && port.IsOpen)
            {
                port.Close();
            }

            port = new SerialPort("COM25", 115200);
            port.Open();

            thread = new Thread(new ThreadStart(writeData));
            thread.Start();
        }

        public void stop()
        {
            if (thread != null && thread.ThreadState != ThreadState.Stopped)
            {
                thread.Abort();
            }

            if (port != null && port.IsOpen)
            {
                port.Close();
            }
        }

        public void setDmxValue(int channel, byte value)
        {
            if (buffer != null)
            {
                if (channel < buffer.Length)
                {
                    buffer[channel] = value;
                }
            }
        }

        public void writeData()
        {
            while (true)
            {
                char[] end_char = { '\n' };

                string raw = string.Concat(buffer.Select(x => x.ToString("000")));
                raw = raw.Remove(raw.Length - 1, 1);
                string b64 = Base64Encode(raw);

                port.Write(b64);
                port.Write(end_char, 0, 1);

                Thread.Sleep(20);
            }
        }

        public static string Base64Encode(string plainText)
        {
            var plainTextBytes = System.Text.Encoding.UTF8.GetBytes(plainText);
            return System.Convert.ToBase64String(plainTextBytes);
        }

        public void WriteArray(byte[] buffer)
        {
            for (int i = 0; i < buffer.Length; i++)
            {
                Console.Write(buffer[i]);
                Console.Write(" ");
            }
        }
    }
}

\end{lstlisting}


\begin{lstlisting}[language=C++]
using System;
using System.IO;
using System.Net;
using System.Net.Sockets;
using System.Security.AccessControl;
using System.Security.Principal;
using System.Text;
using System.Windows.Input;

namespace DMXServer
{
    internal class Program
    {

        static void Main(string[] args)
        {
            int num_leds = 8;
            int num_channels = 4;
            //TcpListener server = null;
            ArduinoDMX arduino_dmx = new ArduinoDMX();
            //OpenDMX open_dmx = new OpenDMX();

            arduino_dmx.start();
            //Int32 port = 13000;
            //IPAddress localAddr = IPAddress.Parse("127.0.0.1");

            //server = new TcpListener(localAddr, port);
            //server.Start();

            //TcpClient client = server.AcceptTcpClient();
            //Console.WriteLine("CLIENT CONNECTED");
            //NetworkStream stream = client.GetStream();

            //while (!client.Connected) ;
            /*
            while (client.Connected)
            {
                int buf_len;
                byte[] buf = new byte[1024];

                if ((buf_len = stream.Read(buf, 0, buf.Length)) > 0)
                {
                    string str = Encoding.UTF8.GetString(buf, 0, buf_len).ToString();
                    int channel = int.Parse(str.Split(new char[] { '=' })[0]);
                    byte value = byte.Parse(str.Split(new char[] { '=' })[1]);
                    arduino_dmx.setDmxValue(channel, value);
                }

            }
            */

            for (int i = 0; i < (num_leds * num_channels); i += num_channels) { arduino_dmx.setDmxValue(i, 200); }

            while (true)
            {
                char key = Console.ReadKey().KeyChar;
                if (key == 'u') { for (int i = 0; i < (num_leds * num_channels); i += num_channels) { arduino_dmx.setDmxValue(1 + i, 255); } }
                if (key == 'j') { for (int i = 0; i < (num_leds * num_channels); i += num_channels) { arduino_dmx.setDmxValue(1 + i, 200); } }
                if (key == 'n') { for (int i = 0; i < (num_leds * num_channels); i += num_channels) { arduino_dmx.setDmxValue(1 + i, 100); } }
                if (key == 'k') { for (int i = 0; i < (num_leds * num_channels); i += num_channels) { arduino_dmx.setDmxValue(1 + i, 0); } }
                if (key == 'r') { for (int i = 0; i < (num_leds * num_channels); i += num_channels) { arduino_dmx.setDmxValue(3 + i, 170); } }
                if (key == 'f') { for (int i = 0; i < (num_leds * num_channels); i += num_channels) { arduino_dmx.setDmxValue(3 + i, 80); } }
                if (key == 'v') { for (int i = 0; i < (num_leds * num_channels); i += num_channels) { arduino_dmx.setDmxValue(3 + i, 30); } }
                if (key == 'd') { for (int i = 0; i < (num_leds * num_channels); i += num_channels) { arduino_dmx.setDmxValue(3 + i, 0); } }
            }

            arduino_dmx.stop();
        }
    }
}

\end{lstlisting}

\chapter{MATLAB TCP/IP Receiver Code}\label{appendix:matlab_receiver_code}

\begin{lstlisting}[language=MATLAB]
clear server

server = tcpserver("0.0.0.0", 2000);
configureTerminator(server, "LF");
configureCallback(server, "terminator", @server_callback);

function server_callback(src, ~)
    disp("-----------------------")
    base64 = readline(src);
    decoded = matlab.net.base64decode(char(base64));
    bytes = uint8(transpose(decoded));
end
\end{lstlisting}

\chapter{ESP32 Transmitter Code}\label{appendix:esp_code}

\begin{lstlisting}[language=C++]
#include <WiFi.h>
#include <ESPmDNS.h>
#include <WiFiUdp.h>
#include <ArduinoOTA.h>
#include <ESP32DMASPISlave.h>

#include <string.h>
#include <stdint.h>
#include "util.h"

#define SERVER_IP "10.1.1.187"
#define SERVER_PORT 2000
#define BUFFER_LENGTH 128

uint8_t* buffer;

ESP32DMASPI::Slave slave;
WiFiClient client;

constexpr uint8_t CORE_TASK_SPI_SLAVE {0};
constexpr uint8_t CORE_TASK_PROCESS_BUFFER {0};

static TaskHandle_t task_handle_wait_spi = 0;
static TaskHandle_t task_handle_process_buffer = 0;

void task_wait_spi(void* pvParameters) {
  while (1) {
    ulTaskNotifyTake(pdTRUE, portMAX_DELAY);
    slave.wait(buffer, BUFFER_LENGTH);
    xTaskNotifyGive(task_handle_process_buffer);
  }
}

void task_process_buffer(void* pvParameters) {
  while (1) {
    ulTaskNotifyTake(pdTRUE, portMAX_DELAY);
    if (client.connected()) { client.write(buffer, slave.available()); }
    slave.pop();
    xTaskNotifyGive(task_handle_wait_spi);
  }
}

void setup() {
  Serial.begin(115200);
  Serial.println("Booting");

  buffer = slave.allocDMABuffer(BUFFER_LENGTH);
  delay(1000);

  slave.setDataMode(SPI_MODE0);
  slave.setMaxTransferSize(BUFFER_LENGTH);
  slave.begin(HSPI);

  xTaskCreatePinnedToCore(task_wait_spi, "task_wait_spi", 2048, NULL, 2, &task_handle_wait_spi, CORE_TASK_SPI_SLAVE);
  xTaskNotifyGive(task_handle_wait_spi);
  xTaskCreatePinnedToCore(task_process_buffer, "task_process_buffer", 2048, NULL, 2, &task_handle_process_buffer, CORE_TASK_PROCESS_BUFFER);

  WiFi.mode(WIFI_STA);
  WiFi.begin(SSID, PASS);

  while (WiFi.waitForConnectResult() != WL_CONNECTED) {
    Serial.printf("Connection Failed! Rebooting... \r\n");
    delay(5000);
    ESP.restart();
  }

  OTA_setup();
  ArduinoOTA.begin();
}

void loop() {
  ArduinoOTA.handle();

  if (!client.connected()) {
    if (client.connect(SERVER_IP, SERVER_PORT)) {
      Serial.printf("Connected to %s on tcp port %u \r\n", SERVER_IP, SERVER_PORT);
    }

    else {
      Serial.printf("Failed to connect to %s on tcp port %u \r\n", SERVER_IP, SERVER_PORT);
      delay(1000);
    }
  }

\end{lstlisting}

\chapter{PIC32 Code}\label{appendix:pic32_code}

\begin{lstlisting}[language=C]

// PIC32MX775F512H Configuration Bit Settings

// 'C' source line config statements

// DEVCFG3
// USERID = No Setting
#pragma config FSRSSEL = PRIORITY_7     // SRS Select (SRS Priority 7)
#pragma config FMIIEN = OFF             // Ethernet RMII/MII Enable (RMII Enabled)
#pragma config FETHIO = OFF             // Ethernet I/O Pin Select (Alternate Ethernet I/O)
#pragma config FCANIO = OFF             // CAN I/O Pin Select (Alternate CAN I/O)
#pragma config FUSBIDIO = ON            // USB USID Selection (Controlled by the USB Module)
#pragma config FVBUSONIO = ON           // USB VBUS ON Selection (Controlled by USB Module)

// DEVCFG2
#pragma config FPLLIDIV = DIV_10        // PLL Input Divider (10x Divider)
#pragma config FPLLMUL = MUL_16         // PLL Multiplier (16x Multiplier)
#pragma config UPLLIDIV = DIV_6         // USB PLL Input Divider (6x Divider)
#pragma config UPLLEN = ON              // USB PLL Enable (Enabled)
#pragma config FPLLODIV = DIV_8         // System PLL Output Clock Divider (PLL Divide by 8)

// DEVCFG1
#pragma config FNOSC = PRIPLL           // Oscillator Selection Bits (Primary Osc w/PLL (XT+,HS+,EC+PLL))
#pragma config FSOSCEN = OFF            // Secondary Oscillator Enable (Disabled)
#pragma config IESO = OFF               // Internal/External Switch Over (Disabled)
#pragma config POSCMOD = HS             // Primary Oscillator Configuration (HS osc mode)
#pragma config OSCIOFNC = OFF           // CLKO Output Signal Active on the OSCO Pin (Disabled)
#pragma config FPBDIV = DIV_1           // Peripheral Clock Divisor (Pb_Clk is Sys_Clk/1)
#pragma config FCKSM = CSDCMD           // Clock Switching and Monitor Selection (Clock Switch Disable, FSCM Disabled)
#pragma config WDTPS = PS1048576        // Watchdog Timer Postscaler (1:1048576)
#pragma config FWDTEN = OFF             // Watchdog Timer Enable (WDT Disabled (SWDTEN Bit Controls))

// DEVCFG0
#pragma config DEBUG = OFF              // Background Debugger Enable (Debugger is disabled)
#pragma config ICESEL = ICS_PGx1        // ICE/ICD Comm Channel Select (ICE EMUC1/EMUD1 pins shared with PGC1/PGD1)
#pragma config PWP = OFF                // Program Flash Write Protect (Disable)
#pragma config BWP = OFF                // Boot Flash Write Protect bit (Protection Disabled)
#pragma config CP = OFF                 // Code Protect (Protection Disabled)

// #pragma config statements should precede project file includes.
// Use project enums instead of #define for ON and OFF.

#include <xc.h>
#include "user.h"

int main (void) {
    init();

    while (1) {
        run();
    }

    // Should never reach this
    return 0;
}

\end{lstlisting}


\begin{lstlisting}[language=C]

#ifndef _SINE_H
#define _SINE_H

#ifdef __cplusplus
extern "C" {
#endif

void init(void);
void run(void);

#ifdef __cplusplus
}
#endif

#endif /* _SINE_H */
#ifndef _SINE_H
#define _SINE_H

#ifdef __cplusplus
extern "C" {
#endif

void init(void);
void run(void);

#ifdef __cplusplus
}
#endif

#endif /* _SINE_H */

\end{lstlisting}

\begin{lstlisting}[language=C]

#include <xc.h>
#include <stdint.h>
#include <string.h>

#include "user.h"
#include "util.h"
#include "ESP32.h"
#include "ADS1294R.h"

struct {
    uint32_t ECG:16;
    uint32_t RSP:16;
    uint32_t EMG:16;
    uint32_t BPM:16;
} packet;

ChannelData ch;

void init() {
//    ADC_init();
    ESP32_init();
    ADS1294R_init();
}

void run() {
//    if (data_ready()) {
//        read_data(&ch);
//        debug(
//            "HEADER: 0x%06X \n"
//            "CH1: %u \n"
//            "CH2: %u \n"
//            "CH3: %u \n"
//            "CH4: %u \n",
//            ch.HEAD, ch.CH1, ch.CH2, ch.CH3, ch.CH4
//        );
//    }
    debug("A");
//    delay(500);
}
\end{lstlisting}

\begin{lstlisting}[language=C]

#ifndef _UTIL_H_
#define _UTIL_H_

#include <xc.h>

// Print 8-bit binary number using printf
// usage: debug("NUM: "BYTE_TO_BINARY_PATTERN, BYTE_TO_BINARY(binary_number));
#define BYTE_TO_BINARY_PATTERN "%c%c%c%c%c%c%c%c"
#define BYTE_TO_BINARY(byte)  \
  ((byte) & 0x80 ? '1' : '0'), \
  ((byte) & 0x40 ? '1' : '0'), \
  ((byte) & 0x20 ? '1' : '0'), \
  ((byte) & 0x10 ? '1' : '0'), \
  ((byte) & 0x08 ? '1' : '0'), \
  ((byte) & 0x04 ? '1' : '0'), \
  ((byte) & 0x02 ? '1' : '0'), \
  ((byte) & 0x01 ? '1' : '0')

// SYS_FREQ = CRYSTAL_FREQ / 10 * 16 / 8
//#define SYS_FREQ 5000000

// May not be exactly actually since scope says different but close enough
#define DELAY_CONST 2.5

void delay_us(unsigned int us) {
    // Convert microseconds us into how many clock ticks it will take
    // Doing this causes an overflow I think... better to precalc and put in magic number
//    us *= SYS_FREQ / 1000000 / 2;   // Core Timer updates every 2 ticks
    us *= DELAY_CONST;
    _CP0_SET_COUNT(0);              // Set Core Timer count to 0
    while (us > _CP0_GET_COUNT());  // Wait until Core Timer count reaches the number we calculated earlier
}

void delay_ms(int ms) {
    delay_us(ms * 1000);
}

void delay(int ms) {
    delay_ms(ms);
}

void ADC_init() {
    AD1CON1bits.ADSIDL = 0;
    AD1CON1bits.SIDL = 0;
    AD1CON1bits.ASAM = 1;   // auto sampling
    AD1CON1bits.CLRASAM = 0; // overwrite buffer
    AD1CON1bits.FORM = 0b000; // integer 16-bit output
    AD1CON1bits.SSRC = 0b111; // auto convert
    AD1CON1bits.ADON = 1;
    AD1CON1bits.ON = 1;
    AD1CON1bits.SAMP = 1;

    AD1CHSbits.CH0SA = 0b1111;
    AD1CHSbits.CH0NA = 0;
    AD1CHSbits.CH0SB = 0b0000;
    AD1CHSbits.CH0NB = 0;
}
#endif // _UTIL_H_
\end{lstlisting}

\begin{lstlisting}[language=C]

#ifndef _ESP32_H_
#define _ESP32_H_

#include <xc.h>
#include <stdint.h>
#include <stdio.h>
#include <stdarg.h>
#include <string.h>

uint32_t ESP32_SPI_write(uint32_t data) {
    // Low-level SPI driver
    SPI2BUF = data;                 // Place data we want to send in SPI buffer
    while(!SPI2STATbits.SPITBE);    // Wait until sent status bit is cleared
    uint32_t read = SPI2BUF;        // Read data from buffer to clear it

    delay_us(5000);
    return read;
}

void ESP32_SPI_write_4byte(uint8_t b1, uint8_t b2, uint8_t b3, uint8_t b4) {
    uint32_t word = ((uint32_t)b1 << 24) | ((uint32_t)b2 << 16) | ((uint32_t)b3 << 8) | (uint32_t)b4;
    ESP32_SPI_write(word);
}

void ESP32_SPI_write_byte(uint8_t data) {
    ESP32_SPI_write_4byte(data, 0, 0, 0);
}

void ESP32_SPI_write_array(uint8_t *array, size_t len) {
    for (size_t i = 0; i < len; i++) {
        ESP32_SPI_write_byte(array[i]);
    }
}

void write_packet(uint8_t* buf, size_t len) {
    uint8_t mod_table[] = {0, 2, 1};
    char encoding_table[] = {   'A', 'B', 'C', 'D', 'E', 'F', 'G', 'H',
                                'I', 'J', 'K', 'L', 'M', 'N', 'O', 'P',
                                'Q', 'R', 'S', 'T', 'U', 'V', 'W', 'X',
                                'Y', 'Z', 'a', 'b', 'c', 'd', 'e', 'f',
                                'g', 'h', 'i', 'j', 'k', 'l', 'm', 'n',
                                'o', 'p', 'q', 'r', 's', 't', 'u', 'v',
                                'w', 'x', 'y', 'z', '0', '1', '2', '3',
                                '4', '5', '6', '7', '8', '9', '+', '/'
    };

    size_t output_length = 4 * ((len + 2) / 3);
    char encoded_data[output_length];

    for (int i = 0, j = 0; i < len;) {
        uint32_t octet_a = i < len ? buf[i++] : 0;
        uint32_t octet_b = i < len ? buf[i++] : 0;
        uint32_t octet_c = i < len ? buf[i++] : 0;

        uint32_t triple = (octet_a << 0x10) + (octet_b << 0x08) + octet_c;

        encoded_data[j++] = encoding_table[(triple >> 3 * 6) & 0x3F];
        encoded_data[j++] = encoding_table[(triple >> 2 * 6) & 0x3F];
        encoded_data[j++] = encoding_table[(triple >> 1 * 6) & 0x3F];
        encoded_data[j++] = encoding_table[(triple >> 0 * 6) & 0x3F];
    }

    for (int i = 0; i < mod_table[len % 3]; i++) {
        encoded_data[output_length - 1 - i] = '=';
    }

    ESP32_SPI_write_array(encoded_data, output_length);
    ESP32_SPI_write_byte('\n');
}

void debug(const char *fmt, ...) {
    va_list args;
    char str[1024];

    va_start(args, fmt);
    vsprintf(str, fmt, args);
    va_end(args);

    write_packet(str, strlen(str));
}

void ESP32_IO_init() {
    TRISBbits.TRISB2 = 0;       // Set ESP32 EN pin as output
    PORTBbits.RB2 = 1;          // Set ESP32 EN pin high
}

void ESP32_SPI_init() {
    SPI2CONbits.ON = 0;         // Turn off SPI2 before configuring
    SPI2CONbits.FRMEN = 0;      // Framed SPI Support (SS pin used)
    SPI2CONbits.MSSEN = 1;      // Slave Select Enable (SS driven during transmission)
    SPI2CONbits.ENHBUF = 0;     // Enhanced Buffer Enable (disable enhanced buffer)
    SPI2CONbits.SIDL = 1;       // Stop in Idle Mode
    SPI2CONbits.DISSDO = 0;     // Disable SDOx (pin is controlled by this module)
    SPI2CONbits.MODE32 = 1;     // Use 32-bit mode
    SPI2CONbits.MODE16 = 0;     // Do not use 16-bit mode
    SPI2CONbits.SMP = 0;        // Input data is sampled at the end of the clock signal
    SPI2CONbits.CKE = 1;        // Data is shifted out/in on transition from idle (high) state to active (low) state
    SPI2CONbits.SSEN = 1;       // Slave Select Enable (SS pin used by module)
    SPI2CONbits.CKP = 0;        // Clock Polarity Select (clock signal is active low, idle state is high)
    SPI2CONbits.MSTEN = 1;      // Master Mode Enable
    SPI2CONbits.STXISEL = 0b01; // SPI Transmit Buffer Empty Interrupt Mode (generated when the buffer is completely empty)
    SPI2CONbits.SRXISEL = 0b11; // SPI Receive Buffer Full Interrupt Mode (generated when the buffer is full)
    SPI2BRG = 50;
    SPI2CONbits.ON = 1;         // Configuration is done, turn on SPI2 peripheral
}

void ESP32_init() {
    ESP32_IO_init();
    ESP32_SPI_init();
}

#endif /* _ESP32_H_ */
\end{lstlisting}

\begin{lstlisting}[language=C]

#ifndef _ADS1294R_H_
#define _ADS1294R_H_

#include <xc.h>
#include <stdint.h>

#include "util.h"
#include "ESP32.h"


/* Pin Mapping */

// Test points
#define TP6_PIN PORTDbits.RD4
#define TP7_PIN PORTDbits.RD5
#define TP8_PIN PORTDbits.RD6

// Controls
#define DRDY_PIN PORTDbits.RD7
#define CLKSEL_PIN PORTDbits.RD8
#define CS_PIN PORTDbits.RD9
#define START_PIN PORTDbits.RD10


/* Register Addresses */

// Device settings (READ-ONLY)
#define ID          0x00

// Global Settings across channels
#define CONFIG1     0x01
#define CONFIG2     0x02
#define CONFIG3     0x03
#define LOFF        0x04

// Channel-specific settings
#define CH1SET      0x05
#define CH2SET      0x06
#define CH3SET      0x07
#define CH4SET      0x08
#define RLD_SENSP   0x0D
#define RLD_SENSN   0x0E
#define LOFF_SENSP  0x0F
#define LOFF_SENSN  0x10
#define LOFF_FLIP   0x11

// Lead-off status registers (READ-ONLY)
#define LOFF_STATP  0x12
#define LOFF_STATN  0x13

// GPIO and other registers
#define GPIO        0x14
#define PACE        0x15
#define RESP        0x16
#define CONFIG4     0x17
#define WCT1        0x18
#define WCT2        0x19


/* SPI Command Definitions */

// System commands
#define WAKEUP      0x02
#define STANDBY     0x04
#define RESET       0x06
#define START       0x08
#define STOP        0x0A

// Data read commands
#define RDATAC      0x10
#define SDATAC      0x11
#define RDATA       0x12


/* Chip info */

// Channel definitions
#define NUMBER_OF_CHANNELS 4
#define BYTES_PER_CHANNEL 3
#define BYTES_TO_READ (NUMBER_OF_CHANNELS * BYTES_PER_CHANNEL)

#define CS_DELAY 0


/* Channel data struct */
typedef struct {
    uint32_t HEAD:24;
    uint32_t CH1:24;
    uint32_t CH2:24;
    uint32_t CH3:24;
    uint32_t CH4:24;
} ChannelData;

#define THREE_BYTE(B1, B2, B3) ((B1 << 16) | (B2 << 8) | B3)


/* Low-level driver */

uint8_t ADS1294R_write(uint8_t data) {
    // Low-level SPI driver
    SPI3BUF = (uint32_t)data;           // Place data we want to send in SPI buffer
    while(!SPI3STATbits.SPITBE);        // Wait until sent status bit is cleared
    return (uint8_t)SPI3BUF;            // Read data from buffer to clear it
}

uint8_t ADS1294R_read() {
    return ADS1294R_write(0x00);
}

void write_cmd(uint8_t cmd) {
    CS_PIN = 0;
    ADS1294R_write(cmd);
    CS_PIN = 1;
}

/* Register drivers */

uint8_t read_register(uint8_t reg) {
    static uint8_t read_register_cmd = 0x20;
    static uint8_t read_register_mask = 0x1F;

    uint8_t first_byte = read_register_cmd | (reg & read_register_mask);
    uint8_t second_byte = 0x00; // only ever read a single register

    CS_PIN = 0;
    ADS1294R_write(first_byte);
    ADS1294R_write(second_byte);
    ADS1294R_read();
    uint8_t ret = ADS1294R_read();
    CS_PIN = 1;

    return ret;
}

void write_register(uint8_t reg, uint8_t data) {
    static uint8_t write_register_cmd = 0x40;
    static uint8_t write_register_mask = 0x1F;

    uint8_t first_byte = write_register_cmd | (reg & write_register_mask);
    uint8_t second_byte = 0x00; // only ever write a single register

    CS_PIN = 0;
    ADS1294R_write(first_byte);
    ADS1294R_write(second_byte);
    ADS1294R_write(data);
    CS_PIN = 1;
}


/* ADS1298RADS1294R init */

void ADS1294R_GPIO_init() {
    // Not sure if any of these work...
    TRISDbits.TRISD4 = 0;       // TP6 as output    -  Pin 52
    TRISDbits.TRISD5 = 0;       // TP7 as output    -  Pin 53
    TRISDbits.TRISD6 = 0;       // TP8 as output    -  Pin 54

    TRISDbits.TRISD7 = 1;       // nDRDY as input   -  Pin 55
    TRISDbits.TRISD8 = 0;       // CLKSEL as output -  Pin 42
    TRISDbits.TRISD9 = 0;       // CS as output     -  Pin 43
    TRISDbits.TRISD10 = 0;      // START as output  -  Pin 44

    TP6_PIN = 0;
    TP7_PIN = 0;
    TP8_PIN = 0;

    CLKSEL_PIN = 0;
    CS_PIN = 1;
    START_PIN = 0;
}

void ADS1294R_SPI_init() {
    SPI3CONbits.ON = 0;         // Turn off SPI2 before configuring
    SPI3CONbits.FRMEN = 0;      // Framed SPI Support (SS pin used)
    SPI3CONbits.MSSEN = 0;      // Slave Select Enable (SS driven during transmission)
    SPI3CONbits.ENHBUF = 0;     // Enhanced Buffer Enable (disable enhanced buffer)
    SPI3CONbits.SIDL = 1;       // Stop in Idle Mode
    SPI3CONbits.DISSDO = 0;     // Disable SDOx (pin is controlled by this module)
    SPI3CONbits.MODE32 = 0;     // Do not use 32-bit mode (8-bit mode)
    SPI3CONbits.MODE16 = 0;     // Do not use 16-bit mode (8-bit mode)

    // SMP = 1; data sampled at end of output time... SMP = 0; data sampled at middle of output time
    SPI3CONbits.SMP = 1;

    // CKE = 1; transition from active to idle... CKE = 0; transition from idle to active
    SPI3CONbits.CKE = 0;

    // CKP = 1; high is idle, low is active... CKP = 0; low is idle, high is active
    SPI3CONbits.CKP = 0;

    SPI3CONbits.SSEN = 0;       // Slave Select Enable (SS pin used by module)
    SPI3CONbits.MSTEN = 1;      // Master Mode Enable
    SPI3CONbits.STXISEL = 0b01; // SPI Transmit Buffer Empty Interrupt Mode (generated when the buffer is completely empty)
    SPI3CONbits.SRXISEL = 0b11; // SPI Receive Buffer Full Interrupt Mode (generated when the buffer is full)

    // SCLK period > 70ns
    // 70ns ~= 14.3MHz
    // F_SCK = 14MHz

    // Library uses 4MHz

    // BRG = (F_PB / 2 * F_SCK) - 1
    // BRG = 1.86
    // BRG >= 2

    SPI3BRG = 4;
    SPI3CONbits.ON = 1;         // Configuration is done, turn on SPI3 peripheral
}


/* Public Functions */

void ADS1294R_init() {
    ADS1294R_GPIO_init();
    ADS1294R_SPI_init();

    // Set CLKSEL pin = 1
    CLKSEL_PIN = 1;
    delay(1);

    write_cmd(RESET);
    delay(1);

    // Send Stop Data Continuous command
    write_cmd(SDATAC);
    delay(1);

//    write_register(GPIO, 0b11110000);

    // Write config registers
    write_register(CONFIG1, 0x86);  // 500 samples/s
    delay(1);
    write_register(CONFIG2, 0x00);  // Test signals disabled
    delay(1);
    write_register(CONFIG3, 0xC0);  // Enable internal reference buffer, no RLD
    delay(1);

    // Send Read Data Continuous command
    START_PIN = 1;
    delay(1);
    write_cmd(START);
    delay(1);
    write_cmd(RDATAC);
    delay(1);
}

void read_data(ChannelData* ch) {
    CS_PIN = 0;

    ADS1294R_read();    // read once to clear out previous buffer
//    ch->HEAD = THREE_BYTE(ADS1294R_read(), ADS1294R_read(), ADS1294R_read());
//    ch->CH1 = THREE_BYTE(ADS1294R_read(), ADS1294R_read(), ADS1294R_read());
//    ch->CH2 = THREE_BYTE(ADS1294R_read(), ADS1294R_read(), ADS1294R_read());
//    ch->CH3 = THREE_BYTE(ADS1294R_read(), ADS1294R_read(), ADS1294R_read());
//    ch->CH4 = THREE_BYTE(ADS1294R_read(), ADS1294R_read(), ADS1294R_read());

    uint8_t h1 = ADS1294R_read();
    uint8_t h2 = ADS1294R_read();
    uint8_t h3 = ADS1294R_read();

    ch->HEAD = (h1 << 16) | (h2 << 8) | h3;

    for (uint8_t i = 0; i < BYTES_TO_READ; i++) {
        ADS1294R_read();
    }

    CS_PIN = 1;
}

uint8_t data_ready() {
    return DRDY_PIN == 0;
}

#endif /* _ADS1294R_H_ */
\end{lstlisting}


\end{document}

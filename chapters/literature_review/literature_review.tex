\section{Literature Review}
\subsection{Introduction}
The use of biosignals to generate music and lighting has been an area of exploration and innovation in the field of live performance technology.
In this literature review, we will examine various approaches and technologies that have been developed in the pursuit of creating music and lighting from biosignals.
We will begin by discussing early attempts at generating music from brainwaves,
and the challenges associated with using electroencephalogram (EEG) signals for live performances.
Then, we will explore the use of other biosignals such as electrocardiogram (ECG), electromyography (EMG), galvanic skin response (GSR), and respiratory rate,
which offer more predictability and discuss why they are better suited to this project.
Finally, we will investigate wireless solutions that enable the integration of biosensors into live performance devices,
and look into existing biosignal based lighting systems, before delving into the Music from Biosignals project;
a project that aims to incorporate biosignals into a wireless platform for live performance.
By reviewing these advancements, we hope to gain insights into the current state of the field and identify areas for further improvement and development.

% SHOULD GO INTO EVERY TYPE OF BIOSIGNAL AND THEN ARRIVE AT THESE SPECIFICALLY
% ALSO SHOULD NOT NECESARILLY ONLY LOOK AT MUSIC/PERFORMANCE FROM THEM
% SHOULD START MUCH WIDER AND THEN ARRIVE WHERE I STARTED HERE

\subsection{Music from Brainwaves}
The earliest attempt at creating music from brain activity is Alvin Lucier's `Music For Solo Performer'~\cite{Lucier:2010}~\cite{Straebel:2014}.
While this system suffers from various technical issues such as high noise,
the fundamental issue with trying to use electroencephalogram (EEG) to generate any kind of performance signal
is that the output of an EEG is not at all rhythmic and contains a lot of randomness.
Additionally, EEG signals have a high potential for artifacting~\cite{Mannan:2018} and require a large number of electrodes~\cite{Piorecky:2019}.
These factors make EEG signals unsuitable for performance in a live setting and thus they will not be incorporated into the project.

\subsection{Rhythmic Signal Approach}
Other biosignals that could be used are electrocardiograms (ECG)~\cite{Afonso:1999}\cite{Pan:1985},
which is a common and painless measurement that is used to monitor the heart~\cite{Mayo:2023}.
Electromyography (EMG)~\cite{Tanaka:2002}\cite{Young:2013},
which measures electrical activity due to the response of muscles~\cite{Hopkins:2023}.
Galvanic skin response (GSR)~\cite{Kurniawan:2013},
which can show the intensity of emotional changes due to the change in conductance of the skin~\cite{Farnsworth:2018}.
And respiratory rate~\cite{Carlos:2011},
which measures the number of breaths per minute~\cite{Hopkins2:2023}.
These signals have a degree of predictability~\cite{Tahiroğlu:2008} which makes them better suited for use in this project.
There are a number of examples of these kinds of signals being incorporated into live performance settings such as
the `Conductor's Jacket' by Nakra and Picard~\cite{Nakra:1998}, and `Stethophone' by Nerness and Fuloria~\cite{Nerness:2019}.
However, these applications are still limited in their flexibility and use due to the restrictive wired nature of these devices.

% WIRELESS SOLUTION SHOULD COVER ALL KINDS OF WIRELESS TRANSMISSION
% RATHER THAN JUST WHAT HAS BEEN DONE IN THE BIOSIGNALS SPACE

\subsection{Wireless Solutions}
Wireless biosensor based performance devices do exist, but there is limited information on them.
Examples of such systems are Yamaha AI's `Transforms a Dancer into a Pianist'~\cite{Yamaha:2018}, and `Emovere' by Jaimovich~\cite{Jaimovich:2016}.
These systems allow performers to elevate their performances by adding an experimental aspect.
However, further exploration could still be done in this space with the addition of lighting control.

\subsection{Biosignal-based Lighting Control}
There is limited activity in controlling lighting using biosignals.
The most relevant research available is Wang's EMG-based Interactive Control Scheme for Stage Lighting\cite{Wang:2022}.
This project uses EMG signals to control lighting in a live performance environment.
However, this project focuses on providing specific control of stage lighting using gestures.
For these gestures to work, the feature extraction algorithm of the device needs to be aware of specific gestures, which are predetermined by the developer.
This limits possible areas of creative exploration, as lighting compositions are predetermined rather than being `found' by the performer.
Thus, there is still room for further developments in this area.

\subsection{The Music from Biosignals Project}
The music from biosignals project has been an ongoing project that attempts to develop and integrate these previously mentioned gaps in literature.
The project has developed an on-body device that acquires biosensors and allows them to be wirelessly transmitted to a PC for processing~\cite{Pierro:2019}\cite{Tran:2022}.
Additionally, software developed in MATLAB has been developed that processes the incoming signals and generates music in real-time~\cite{Chen:2016}\cite{Nicholls:2019}.
Prior to the work set out in this thesis, the hardware and software elements of this project had been separate and not yet integrated.
This leaves potential future work open in connecting both sides of the project to create one cohesive whole.
Additionally, there has only ever been a musical aspect to the project, allowing further exploration into how lighting could improve the system.

\subsection{Conclusion}
In conclusion, the exploration of generating music from biosignals has seen significant progress in recent years.
While early attempts using brainwaves faced challenges due to their non-rhythmic and unpredictable nature,
other biosignals such as ECG, EMG, GSR, and respiratory rate have shown promise in generating more predictable and rhythmic signals for use in live performance environments.
While a number of projects have incorporated these signals, there is still room in the literature for exploration in applying these signals to lighting systems, in a way that does not limit the creativity of the performer.

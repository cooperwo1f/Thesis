\section{Literature Review}
\subsection{Introduction}
The use of biosignals to generate music has been an area of exploration and innovation in the field of music technology.
In this literature review, we will examine various approaches and technologies that have been developed in the pursuit of creating music from biosignals.
We will begin by discussing early attempts at generating music from brainwaves
and the challenges associated with using electroencephalogram (EEG) signals for live performances.
Then, we will explore the use of other biosignals such as electrocardiogram (ECG), electromyography (EMG), galvanic skin response (GSR), and respiratory rate,
which offer more predictability and are better suited to this project.
Finally, we will investigate wireless solutions that enable the integration of biosensors into live performance devices and
delve into the Music from Biosignals project; a project that aims to incorporate biosignals into a wireless platform for live performance.
By reviewing these advancements, we hope to gain insights into the current state of the field and identify areas for further improvement and development.

\subsection{Music from Brainwaves}
The earliest attempt at creating music from brain activity is Alvin Lucier's `Music For Solo Performer'~\cite{Lucier:2010}~\cite{Straebel:2014}.
While this system suffers from various technical issues such as high noise,
the fundamental issue with trying to use electroencephalogram (EEG) to generate any kind of performance signal
is that the output of an EEG is not at all rhythmic and contains a lot of randomness.
Additionally, EEG signals have a high potential for artifacting~\cite{Mannan:2018} and require a large number of electrodes~\cite{Piorecky:2019}.
These factors make EEG signals unsuitable for performance in a live setting and thus they will not be incorporated into the project.

\subsection{Rhythmic Signal Approach}
Other biosignals that could be used are electrocardiogram (ECG)~\cite{Afonso:1999}~\cite{Pan:1985},
electromyography (EMG)~\cite{Tanaka:2002}~\cite{Young:2013}, galvanic skin response (GSR)~\cite{Kurniawan:2013}, and respiratory rate~\cite{Carlos:2011}.
These signals have a degree of predictability~\cite{Tahiroğlu:2008} which makes them better suited for use in this project.
There are a number of examples of these kinds of signals being incorporated into live performance settings such as
the `Conductor's Jacket' by Nakra and Picard~\cite{Nakra:1998}, and `Stethophone' by Nerness and Fuloria~\cite{Nerness:2019}.
However, these applications are still limited in their flexibility and use due to the wired nature of these devices.

\subsection{Wireless Solutions}
Wireless biosensor based performance devices do exist.
Examples of such systems are Yamaha AI's `Transforms a Dancer into a Pianist'~\cite{Yamaha:2018}, and `Emovere' by Jaimovich~\cite{Jaimovich:2016}.
There is not much documentation for these systems as they are still in use.
But, from the small number of performance recordings of these systems it is clear that they are intended for experimental music.
Therefore, further improvements (\textbf{SUCH AS}) on these devices can still be made in order to garner mass audience appeal through more conventional mainstream music generation.

\subsection{The Music from Biosignals Project}
\textbf{MORE DETAIL WOULD BE BETTER}

The music from biosignals project has been ongoing for several years and attempts to integrate previously mentioned improvements.
The project has developed an on-body device that acquires biosensors and allows them to be wirelessly transmitted to a PC for processing~\cite{Pierro:2019}~\cite{Tran:2022}.
Additionally, software developed in MATLAB has been developed that processes the incoming signals and generates music in real-time~\cite{Chen:2016}~\cite{Nicholls:2019}.
Previously, the hardware and software design of the system has been separate and the two parts are yet to be integrated.
This leaves potential future work open in connecting both sides of the project to create one cohesive whole.

\subsection{Conclusion}
In conclusion, the exploration of generating music from biosignals has seen significant progress in recent years.
While early attempts using brainwaves faced challenges due to their non-rhythmic and unpredictable nature,
other biosignals such as ECG, EMG, GSR, and respiratory rate have shown promise in generating more predictable and rhythmic signals for use in live performance environments.
The development of wireless solutions has facilitated the incorporation of biosensors into live performance settings,
although further improvements are needed to enhance their mass audience appeal.
The Music from Biosignals project aims to incorporate these changes to create a device that can be used in a variety of live performance settings.
By continuing to explore and refine the use of biosignals in music generation, we can unlock new possibilities for artistic expression and interactive musical experiences.

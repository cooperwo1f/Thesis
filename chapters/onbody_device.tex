\chapter{On-Body Device}
The hardware I was given was in the process of being programmed.
It consists of two microcontrollers, a PIC32 and a ESP32.
The ESP32 had been programmed intermittently
and the PIC32 had been successfully programmed once in order to enable the PIC32.
I sucessfully programmed the ESP32 and enabled Over The Air (OTA) programming.
I also implemented TCP/IP communication between the ESP32 and MATLAB.
Craig Dawson shared code that allowed the PIC32 to be programmed.
I had previously had issues programming it
because the device used on the schematics is different
to what has actually been put on the board.
The code Craig has given me allows for simple programming,
but some of the settings for the hardware have not been set correctly
and the board cannot be put into debug mode.

The process that has been followed to get the hardware working
is to first program the ESP to allow it to be programmed consistently and conveniently.
Then, communication with MATLAB was established using TCP/IP.
This allows for data to be visuallised
which can help with debugging the rest of the hardware.
The PIC was then programmed to allow debugging,
in order to do this the ICE/ICD Comm Channel Select needed to be changed to PGC1/PGD1,
and DEBUG needed to be enabled.
Finally, the PIC was programmed to communicate with the ESP
and the whole chain was tested to verify communication between the PIC and MATLAB.

Difficulty in that Arduino only provides code for SPI master mode.
While ESP (programmed using Arduino) in this configuration makes more sense as the slave.
Somewhat gotten around it using external libraries.
Also difficulty in that anything that halts the ESP loop causes the OTA handle to stop.
This means that any bugs that cause an infinite loop also make the ESP unprogrammable.

Also working to keep all code general.
So for instance can keep MATLAB code running
for entire sequence of testing without needing to change anything.

\section{ESP32}
\subsection{Power}
The brownout detector keeps tripping.
The power distribution of the board is bad and there is voltage drops across traces
Measuring 2.6V (3.3V required) when supplying the board from a single power supply
(current limit is 2x what is required)

Almost 2.7V when adding a secondary ground connection
Closer to 2.8V with additional voltage connection
2.9V with a 3rd voltage connection

\subsection{Firewall}
For the device to connect to the PC the firewall setting need to be correctly set.

\subsection{Programming}
Programming took ages and would fail often.

\subsection{Transmission}
Before I was filling a buffer from the incoming data and waiting until I received
a newline before sending the entire buffer over TCP/IP all at once.
Now I receive data and immediately send, provided client.connected()
I use slave.available() to specify the length of data to be sent.

This is better because it keeps the TCP/IP transmitter simple and also allows all
of the packet metadata to be changeable independently of the ESP.

Currently having a number of IO issues with the PIC32.
Seems like no matter how I set the IO register direction, or how I set
the PORT or LATCH bits, the IO pin never changes.
I appear to be able to correctly read from the IO port although this
hasn't been fully tested I am just able to see ports that should be
inputs driven high.

It may be worth setting those to outputs and see if I still read them
as inputs because it seems like there is no pin assertion happening.

Setting the TRIS bit and then writing the PORTx bit high is all that
needs to be done.
I can verify this because the ESP32 needs its enable pin to be pulled
high and that is done sucesfully using `TRISBbits.TRISB2 = 0;' and
`PORTBbits.RB2 = 1;'
(testing this by setting `PORTBbits.RB2 = 0' resulted in the ESP
shutting down)
It just seems that no matter what I do, I cannot get signal on any of
the test points. I'm trying to determine if it's an issue with the
program or if the pinout is incorrect or if the routing was not done
properly or some other issue like maybe my scope isn't referenced to
the correct ground and the voltage that is present is actually not
being measured.

I'm thinking a reasonable way to proceed would be to continuity test
a specific pin and 100\% verify the port and pin before checking it just
with the multimeter to determine if voltage is present. Then, once that
has be sorted move on to using the scope.
What do we know:

    - Pin outputs are set using TRISx and PORTx.
    - The enable pin for the ESP is using one of these pins
        - Although all that I've really been seeing
        is the brownout detector being triggered
        so maybe the ESP not working is unrelated to the
        tampering I was doing with that pin...
    - I am not able to see changes on any of the test points
        - This is really strange because I am setting the entire port
        worth of pins so I don't see how they could be getting missed.

I think it's worth investigating the ESP enable pin a little closer.
Maybe check voltage with a multimeter since it should be pretty easy.
Then, if there is voltage attempt to remove and see what the ESP does
(when properly powered using external supply as to not trip
brownout detection).
Then continue to different pin on same port, verify that is also working.
Finally, attempt changing port and test off pin of controller.
Continuity test between pin and test point then check test point.
If all correct we know 100% that the issue is with the scope.

Might have something to do with the brownout detector that the ESP is constantly triggering.
Think this is because the board requires more power to perform specific actions.
For instance, I think the PIC debugging requires a bit of power because of the speed,
and when it happens the ESP voltage drops low enough that it causes a brownout.
My solution to this is to decrease the clock speed of the PIC since the 80MHz is much
faster than what we actually need the system to ever run at considering the requirements.

Slowing the clock speed did not fully solve the issue and sometime the brownout detector
is tripped again...
Attempting to solve by plugging in a short USB into port of board.
Also need to enable ICD 3 powering of board.

\section{PIC32}
\subsection{Packet}
%\lstinputlisting[language=C]{chapters/research/package_test/simple.c}
The packet is made up of a struct containing bit length defs
for each of data types we will be sending.

This struct is then treated as an array and encoded using base64
(decision matrix pending).

The encoded data is then sent via SPI to the ESP32 which acts
as a SPI slave that simply receives any amount of data
(less than 1024 bytes long) and transmits it once it receives
the end of line character (`\\n').
This character is hard coded but it also required to be present
for the MATLAB code to function so it is consistent across both.

The challenge now is that the data needs to be decoded in MATLAB.
The following code can be used in a callback function to convert
the received encoded data into an array of byte values.

\begin{lstlisting}[language=MATLAB]
base64 = readline(src);
decoded = transpose(matlab.net.base64decode(char(base64)));
\end{lstlisting}

The decoded data can then be converted into the correct format
using the following code

\begin{lstlisting}[language=MATLAB]
x = uint8(decoded(9:12));
disp(typecast(x, 'uint32'))
\end{lstlisting}

The issue here is that there is a lot of hardcoded values
in both the index of the data and the conversion format.
One solution is to remove all bit length defs and send everything
as 32-bit words. The tradeoff is that it is potentially quite wasteful.
For instance, if we only needed to send 8-bits, we are essentially
sending 24-bits of unnecessary data that will all be 0s.

We ideally want to send a header at the start of transmission that
contains all the information regarding size of data, etc.
Then send a packet identifier so the system knows what data to expect.

What would be good is to have all the identifiers be bitwise OR'd
with each other so that the MATLAB code can just look at the identifier
bit and can tell which pieces of data are present.
This is better than the alternative of having seperate identifiers for
every combination of data because then we can just define everything once.
For example our header could look like the following:

\begin{lstlisting}[language=C]
IDENTIFIER = 0b10011000
\end{lstlisting}

which might correspond to ECG data, RESP data, and EMG data.
The definitions for those data masks could be as follows:

\begin{lstlisting}[language=C]
ECG_MASK = 0b10000000
RSP_MASK = 0b00010000
EMG_MASK = 0b00001000
\end{lstlisting}

We could see from this that the identifier contains the corresponding
masks and use it to extract the data.
We could then say that the highest mask takes prescience.
So, in this example the ECG data would come first since its mask's bit
is the highest in the identifier.

Then, all we need to define is that the mask bit corresponds to x amount
of data. So that once we see a specific mask bit is set we can
automatically align the data.

I think its best to do this first manually by having all the masks
and sizes set. Then, once that is working move to having a config packet
that sets the masks and sizes dynamically.
The goal is to keep the definitions in once place and propagate it
through the system.

Wrote a debug routine to print arbritary text to MATLAB via SPI/TCP.

\begin{lstlisting}[language=C]
void debug(const char *fmt, ...) {
    va_list args;
    char str[1024];

    va_start(args, fmt);
    vsprintf(str, fmt, args);
    va_end(args);

    write_packet(str, strlen(str));
}
\end{lstlisting}

Then can receive in MATLAB using

\begin{lstlisting}[language=MATLAB]
y = char(decoded);
disp(y)
\end{lstlisting}

\subsubsection{SPI}
Currently running into issues where the SPI transmission does not work consistently.
Tested it independently and got it presumably working quite well but for some reason
now that it's been implemented into the rest of the chain it only works if there is
significant delay.

\begin{lstlisting}[language=C]
for (size_t i = 0; i < output_length; i++) {
    char str[128];
    sprintf(str, "%c", encoded_data[i]);
    ESP_write_array(str, strlen(str));
}
\end{lstlisting}

This was originally the only way to get transmission to work.
output length is the length of the base64 encoded string.
I essentially had to write each character seperatly as if it were a string.
This is a problem because the write function that I am using contains a significant delay.

\begin{lstlisting}[language=C]
void ESP_write_array(uint8_t *array, size_t len) {
    for (size_t i = 0; i < len; i += 4) {
        ESP_write_4byte(array[i], array[i+1], array[i+2], array[i+3]);
    }

    if (len % 4 == 1) { ESP_write_4byte(array[len-1], 0, 0, 0); }
    if (len % 4 == 2) { ESP_write_4byte(array[len-1], array[len-2], 0, 0); }
    if (len % 4 == 3) { ESP_write_4byte(array[len-1], array[len-2], array[len-3], 0); }

    delay(10);
}
\end{lstlisting}

There needs to be a delay between each 32-bit word otherwise the transmission fails.
This is beacuse the PIC32 will continually transmit without resetting the chip select line
and the ESP32 needs to reset after each 32-bit word, as far as I can tell this is set.

The idea of the `ESP\_write\_array' function is that it can pack 4 bytes into a signle word
to negate as much of this delay as possible.


\section{ADS1294R}
\subsection{Design Differences}
The chip that was actually used in the design is the ADS1294R!!
This is contrary to what the schematic and PCB design designators say.
The only way I figured this out was to physically inspect the board.
Mostly the same except this one only has 4 channels and so the expected data
to be read is \(24 + 4 \times 24 = 120\) bits.
Also does not contain all the same registers as the bigger brothers.

Other differences are that the ID should be `0b11010000'
Registers `RLD\_SENSP' and `RLD\_SENSN' do not contain bits[4:7]
Registers `LOFF\_SENSP' and `LOFF\_SENSN' do not contain bits[4:7]

As far as I can tell from the datasheet, everything else should be the same.

Also appears to be additional connection from nDRDY to DGND with some capacitors.

Not 100\% sure if this is actually happening though because only image of PCB design
low res screenshot of only top side of board. Very hard to follow traces.
As far as I can tell it seems reasonable that each trace routed from the PIC to the ADS
is correct. Not really possible to check as it is a ZXG package (basically a BGA package).

Also pin 43 of the PIC is connected to DOUT which according to the datasheet of the PIC
is actually the SPI chip select pin not the data input pin.

Might have to write a custom driver that just pulls whatever pin the actual chip select
is connected to low, writes whatever the equivalent SCK pin is high and write/reads
the corresponding SDO and SDI pin manually.

Not sure if any of this is actually managable anyway because the reset switch
is a push button and doesn't appear to have any way of switching it via the PIC.

Can't even really tell where the reset pin goes because it is grounded to the sleve of
the 3.5mm jack?!? Whatever is routed to the switch is on the underside of the board
that I don't have access to.

Currently trying to change all the configurations to see if any combinations work better.
If this doesn't work I can look at the actual altium schematics I got from Kenneth.
Thinking first I'll just try and get the actual ID from the chip
since it is easy to verify.
Going to play around with the driver, add delays between reads and writes and try different
SPI configs to see if anything makes it respond.
Also will try same thing after physically pressing reset switch because maybe it needs
hardware reset.

My scope probe wasn't working because of the x10 setting on the probe was x1.
I actually have managed to get something out of the ID register.

It should be    11010000
It actually is  01100000

This was gotten by setting SPI bits SMP = 0, CKE = 1, CKP = 0, and pressing
the reset button for the chip while using the scope to send a signal at the moment it
would have to be reset in software.
Managed to repeat this at least one more time. Attempting without pressing hardware reset.
Still works without hardware switch being pressed. Not correct but better than nothing.
Only thing that I changed was using the scope to correctly set the delay constant.

Used scope to measure space between pulses then set a specific pulse delay.
I think previously the delay number was overflowing potentially because it was so large.
Potentially the delay was way too short and the timing wasn't correct.

Can get 11000000 if I set SMP = 1
Still missing bit 4 which should be 1 no matter what.
Trying without hardware reset. Still works.

Setting CKP causes ID to be 0, so that should 100\% be reset.

Resetting it and running again causes ID to be 0
Pressing reset while it's running fixes the issue.
Still not what it 100\% should be but it's close.

Increasing the delay amount did not change anything.
Seems like hardware reset can be pressed midrun and it recovers nicely

Without delays it often does not read correctly.
Going to add a single delay in the SPI write function to set max speed.

Going to reduce the delays in the init function. Everything seems to work fine.
Going to try and remove them completely. Seems to work correctly still.
Appears that the singular delay in the SPI write function was enough and nothing else
needs it now. Probably good to do this with the ESP SPI write so that the delays are
only in a single place and can be easily optimised.

Added a substantial amount of extra delay to both to guarantee consistency, can easily
be optimised later.

Changing SPI3BRG as well to see if that makes any kind of difference.
Seems like it can be about any value and it still works the same.
Calculations say it should be greater than 2 so I'll set it to 4 for safety.
However, still works without propogating errors with a value of 1 which
should be too fast.

Wondering if the GPIO init I am doing has an effect. Turning it off causes ID to be 0.
Which is good because it means that I am actually doing something.

Going to go through init function and see what turning various things on/off
causes the system to do.
`CLK\_SEL = 0' causes the chip to stop responding.
Not writing SDATAC doesn't appear to change anything but that could be because I'm not
doing anything that needs registers set.
`START\_PIN = 0' causes chip to stop responding.

Ok now that I have a baseline that I can consistenly get to I'm going to try and do more.


\subsection{Reading Data}
When reading data it doesn't seem like the DRDY pin ever actually goes low.
Even when not sending read data continuous it is always high.
Not really sure if I can trust this though because the pin isn't exposed
So I'm setting a test point high if it is read high by the microcontroller and it may
just be staying high because by the time it can poll again it has already read all the data
and thus the chip is ready to put more data out.

Going to see what happens when I try and read without setting RDATAC.
Ok so issue now is that I am always reading 11000000 even when not reading the ID.

This is really strange to me because it should be shorted so reading 0.
Wonder if that also means that the ID I was reading is not actually correct since it wasn't
exactly what it should have been anyway and it matches what the chip sends continuously.

From my calculations I am reading 15 bytes which is 120 bits which I belive is correct.
If I don't short the input it is still 11000000

Ok so something definitly not right because even once read a few bytes the `DRDY'
pin is still high when it should be low.

It's strange because the DRDY pin is active low, in software I am checking `DRDY\_PIN == 0'
which made me think that maybe the chip just isn't on at all hence why I'm always seeing
`data\_ready()' as true.
But this isn't the case since it becomes false if I hold down the reset pin.
Which means the physical pin is being pulled high when the chip is put into reset which
seems super strange to me because I would assume everything would just go to 0.
At least in this case I can confirm that it is actually working.

It seems wrong that sending the stop data read continuous command doesn't stop the chip
from setting it's data ready pin. Not super logical to me because shouldn't that only be
true when the data is continuously being streamed.

I think the fact I am getting 192 (AKA 11000000) all the time, probably means that all the
register writes I am doing are not actually correct. What I should do I think is go back
to the ID register and attempt to get that to send correctly.

I am not entirely sure where I should start because it seems like this should be working.
I'm going to go back to the SPI config and see what I can maybe do.

Also might be worth looking at software libaries for Arduino to see what they do in say
a simple ID test script.
Maybe there is something here I am missing that is really obvious.

I think it is the power-on timing. Most likely due to hte lack of /RESET pulse since
that was never connected to the PIC.

Considering that, I think it's fair to assume that the timings may still be too short.
Might be worth looking at existing Arduino library for how they setup their SPI.
Also worth looking at schematics to verify there weren't any revision changes with how
the chip RESET is connected (although I doubt there would be).

Looking at presumably functional Arduino library:
clock polarity 0, clock phase 1, output edge rising, data capture falling

What this actually means:
Idles on logic low, data transmission from clock idle to clock active,
data shifted out on rising clock edge, data sampled on falling clock edge

\textbf{ADDITIONALLY}: Data is MSB first with a datarate of 4,000,000 (AKA 4MHz)

Also just realised it doesn't actually matter what I set the baudrate generator to because
the system clock is 5Mhz so it will literally always be able to keep up.


\subsection{Inconsistencies}
Interestingly enough writing START and STOP appears to work.
So the actual writing to SPI seems to be correct.

Very weird because it seems like there are two versions of the design.
The one that got made appears to be older and worse. So not really sure what's going on.
Perhaps this is the beginning of a revision 2?
But I also cannot find the altium files for it, only a single pdf remains.
Shame because the newer one has a bunch of LEDs on it that would make this way easier.

So what we know is that write must work. As writing START and STOP commands have
an effect.
What is currently unknown is if the reading and writing register commands work correctly.
I really don't see how they wouldn't because I've hand checked the bits during debugging.
Really the only thing I can think at this stage is that maybe the PIC is sending in 32-bit
mode. That doesn't really make sense since it's configured to 8-bit mode but maybe it
still sends extra or something (this would be very silly).

I cannot think of anything else to check short of soldering to the pins and checking
the physical signals that are being send both ways. I'm sure if I did that I would be able
to verify it almost immediately... Should definitly bring bodge wire on Tuesday and make
use of the good soldering irons, flux, microscopes, and nice oscilloscopes at uni.

Other than that it could also be the power on cycle needs specific lines held low/high
and I just do not have digital access to those pins.

Or worst case the entire board has been designed so badly that the chip is having weird
power issues. Looking at the altium board files it definitly does not look great.
At least there is a lot to include in ADC challenges for my results.


The status word that I am expecting contains the following
`1100 LOFF\_STATP[7:0] LOFF\_STATN[7:0] GPIO[7:4]'
So maybe the received 11000000 could be the beginning of that status word but then I
am not reading any more or something?

GPIO register looks like `DATA[4:1] CONTROL[4:1]' with 0 being outputs and 1 being inputs
for the control section

So I am not able to write the register. I think at this stage what must be wrong is my
register writing/reading.
I think the actual write commands are working
because if I write sleep, wakeup, start, stop, or reset
the chip has the response I would expect it to have.
I am not able to verify SDATAC or RDATAC because the data I am then reading is wrong.
It also doesn't seem to stop it from giving me data if I send either of them.
Maybe I should try RDATA and try to just do a simple read see if that makes any difference.

Otherwise, I think the write byte works so I don't see why writing the read/write register
command then the address of the register is not working.

It could also be that the SDATAC, RDATAC, and RDATA commands do not work for some reason
This would mean that all my attempts to read and write registers is futile because
the chip is not going to respond to those commands as it boots into RDATAC mode.
But even so the chip does not give any readings, just 192 (AKA 11000000) continously
so I have no idea what that is supposed to mean.


\subsection{Solution}
Ok so the issue actually came from the CS line. The line was being driven active for each
command but because the ADC had multi-byte commands it was expecting the CS line to stay
active for the entire duration.
What was instead happening was the CS line was going inactive
and causing the ADC to reset mid command which meant it was only responding to single byte
command hence why RESET, STANDBY, WAKEUP, etc were working correctly.

Issue now is my driver has an issue because I'm reading all 0s for the
ID but on the scope I can see the response is correct.

Issue was that the SPI buffer holds the previous transmitted value so need to clear it once
before reading again to get actual value.

Also need to make sure to use special `write\_cmd()' function as it contains the neccessary
chip select setting/resetting and it will not respond to commands otherwise.

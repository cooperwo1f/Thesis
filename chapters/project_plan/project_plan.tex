\section{Project Plan}
\subsection{Project Objectives}
For this project to be successful, it is necessary to fulfill various requirements.
The requirements for this project are:

\begin{enumerate}
    \item The system must implement lighting control
    \begin{enumerate}
        \item The protocol for lighting control should be available on at least 80\% of performance based lighting fixtures
        \item The lighting control implementation must be compliant to the protocol
    \end{enumerate}
    \item The system output must operate remotely to the acquired sensor data at a range of at least 30 m
    \begin{enumerate}
        \item The acquisition of sensor data must be detached from the system output
        to allow for more complicated off-body setups
        \item The remote system must be compliant with ISO/IEC 15149-1:2014 or equivalent
        \item The wireless system must have a Packet Error Rate (PER) of no more than 1\%
        \item The wireless system must still be functional in a noisy environment
    \end{enumerate}
    \item The system must generate rhythmic music from the acquired biosignals
    \begin{enumerate}
        \item The musical signals need to be correlated to a consistent beat
    \end{enumerate}
\end{enumerate}

\subsection{Assumptions and Constraints}
Assumptions:
\begin{itemize}
    \item The system will be used in an indoor environment with stable temperature and humidity.
    \item The performers will be able to wear biosignal sensors comfortably during their performance.
    \item The sweat from the performers will be mitigated to avoid short-circuiting the electrodes.
    \item The system will exist in a stable performative environment.
        \begin{itemize}
                \item The various audio visual elements of the performative environment are functional.
                \item The system is being operated by skilled and knowledgeable professionals.
        \end{itemize}
\end{itemize}

Constraints:
\begin{itemize}
    \item The total cost of the system cannot exceed \$600.
    \item The size and weight of the whole system must be compact enough to transport in a standard travel bag.
        \begin{itemize}
                \item The system must weigh less than 20kg.
                \item The system must take up less than 50L.
        \end{itemize}
    \item The size and weight of the on-body subsystem must be wearable for several hours without fatigue.
        \begin{itemize}
                \item The on-body element of the system must weigh less than 5kg.
        \end{itemize}
    \item The system must be compatible with a standard lighting protocol.
\end{itemize}

\subsection{Previous Work}
This project is a continuation of the Music from Biosignals project.
As such, a number of these objectives have already been fulfilled, and some hardware and software has been developed.
The extent of these developments are as follows:

\begin{itemize}
        \item Hardware for an on-body device has been developed.
        \begin{itemize}
                \item The hardware has been successfully powered-on.
                \item Programmer communication with the microcontrollers has been established.
                \item One of the microcontrollers has been connected to WiFi.
        \end{itemize}
        \item Software for the off-body PC has been written.
        \begin{itemize}
                \item a MATLAB script that generates music using a prerecorded ECG biosignal has been developed.
        \end{itemize}
\end{itemize}

\subsection{Scope}
Given this project outline and the previous project work.
The scope of the project can be defined in~\autoref{tab:scope},~\autoref{tab:scope_stretch}, and~\autoref{tab:scope_out}.
This scope contains stretch goals because some of the previous work had not been completed at the start of the project,
as those students were finishing mid-year.
Therefore, the stretch goals exist to allow the necessary project flexibility to account for various outcomes of those student's projects.

\begin{table}[!ht]
    \caption{Project in-scope list}\label{tab:scope}
    \centering
        \begin{tabular}{|l|c|}
        \hline
        \multirow{2}{7em}{On-body Device}     & \cellcolor{green!25}Communication between sub-systems           \\ \cline{2-2}
        ~                                     & \cellcolor{green!25}Basic sensor reading                        \\ \hline \hline
        \multirow{2}{7em}{Lighting Control}   & \cellcolor{green!25}Lighting fixture control                    \\ \cline{2-2}
        ~                                     & \cellcolor{green!25}Test setup                                  \\ \hline \hline
        \multirow{3}{7em}{System Integration} & \cellcolor{green!25}Wireless communication                      \\ \cline{2-2}
        ~                                     & \cellcolor{green!25}Integration with music generation           \\ \cline{2-2}
        ~                                     & \cellcolor{green!25}Integration with lighting generation        \\ \hline
    \end{tabular}

\end{table}

\begin{table}[!ht]
    \caption{Project stretch-goal list}\label{tab:scope_stretch}
    \centering
        \begin{tabular}{|l|c|}
        \hline
        Sensors                               & \cellcolor{orange!25}Wearable sensor design             \\ \hline \hline
        \multirow{3}{7em}{MIDI}               & \cellcolor{orange!25}Compliant MIDI implementation      \\ \cline{2-2}
        ~                                     & \cellcolor{orange!25}MIDI timecode quantisation         \\ \cline{2-2}
        ~                                     & \cellcolor{orange!25}MIDI output timecode               \\ \hline \hline
        \multirow{4}{7em}{System Integration} & \cellcolor{orange!25}Performance application testing    \\ \cline{2-2}
        ~                                     & \cellcolor{orange!25}Bi-directional communication       \\ \cline{2-2}
        ~                                     & \cellcolor{orange!25}Real-time system tunability        \\ \cline{2-2}
        ~                                     & \cellcolor{orange!25}PCB housing design for main board  \\ \hline
    \end{tabular}

\end{table}

\begin{table}[!ht]
    \caption{Project out-of-scope list}\label{tab:scope_out}
    \centering
        \begin{tabular}{|l|c|}
        \hline
        Sensors                               & \cellcolor{red!25}Sensors as independent systems        \\ \hline \hline
        \multirow{2}{7em}{MIDI}               & \cellcolor{red!25}Wireless MIDI                         \\ \cline{2-2}
        ~                                     & \cellcolor{red!25}MIDI threshold points                 \\ \hline \hline
        Lighting Control                      & \cellcolor{red!25}Lighting controller input support     \\ \hline \hline
        \multirow{2}{7em}{System Integration} & \cellcolor{red!25}Cableless system                      \\ \cline{2-2}
        ~                                     & \cellcolor{red!25}User interface for configuration      \\ \hline
    \end{tabular}

\end{table}

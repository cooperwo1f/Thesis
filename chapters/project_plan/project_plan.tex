\section{Project Plan}
\subsection{Project Objectives}
For this project to be successful, it is necessary to fulfill various requirements.
The requirements for this project are:

\begin{enumerate}
    \item The system must be composed of inexpensive parts
    \begin{enumerate}
        \item Sensors must fall within the allocated budget
        \item PCBs must use off the shelf, easily sourced components
        \item Components must be easily replaceable for minimal cost if something were to fail
    \end{enumerate}
    \item The system output must operate remotely to the acquired sensor data at a range of at least 30m
    \begin{enumerate}
        \item The acquisition of sensor data must be completely detached from the system output
        to allow for more complicated off-body setups
        \item The remote system must be compliant with ISO/IEC 15149-1:2014 or equivalent
        \item The wireless system must have a Packet Error Rate (PER) of no more than 1\%
        \item The wireless system must still be functional in a noisy environment
    \end{enumerate}
    \item The system must output MIDI messages
    \begin{enumerate}
        \item The MIDI output port must be IEC 63035:2017 compliant
        \item The system must also provide MIDI over USB for greater compatibility with newer devices
    \end{enumerate}
    \item The system must incorporate lighting control
\end{enumerate}

\subsection{Assumptions and Constraints}
Assumptions:
\begin{itemize}
    \item The users of the system have basic knowledge of music theory and performance.
    \item The system will be used in conjunction with a lighting console that maps all the fixture addresses correctly.
    \item The system will be used in an indoor environment with stable temperature and humidity.
    \item The performers will be able to wear biosignal sensors comfortably during their performance.
    \item The sweat from the performers will be mitigated to avoid short-circuiting the electrodes.
    \item The system will be operated by a skilled technician during live performances.
\end{itemize}

Constraints:
\begin{itemize}
    \item The total cost of the system cannot exceed \$600.
    \item The size and weight of the whole system must be compact enough to transport in a standard travel bag.
    \item The size and weight of the on-body subsystem must be wearable for several hours without fatigue.
    \item The system must be compatible with standard MIDI interfaces and protocols.
\end{itemize}

The given constraints are cost, size, weight, and compatibility.
The cost is a constraint because the budget of the project is separate to the \$600 allocated limit.
So, if the project goes over budget it may continue, but if it goes past the allocated limit, the project will not be able to continue.
The size and weight are constraints because the project needs to be portable enough to effectively use.
The system compatibility is a constraint because it is required for the system to correctly operate with other unknown systems.

\subsection{Scope}
In order to mitigate the risk of other concurrent projects running over scope, additional stretch goals have
been added to the scope of this project. The stretch goals allow for increasing in scope
while still strictly maintaining an achievable scope for this project.
A breakdown of the scope of this project can be found in \autoref{tab:scope}, \autoref{tab:scope_stretch}, and \autoref{tab:scope_out}.

\begin{table}[!ht]
    \caption{Project in-scope list}\label{tab:scope}
    \centering
        \begin{tabular}{|l|c|}
        \hline
        \multirow{2}{7em}{Sensors}            & \cellcolor{green!25}Clothing based biosignal acquisition \\ \cline{2-2}
        ~                                     & \cellcolor{green!25}Sensor gain and filtering            \\ \hline \hline
        \multirow{3}{7em}{System Integration} & \cellcolor{green!25}Wireless communication               \\ \cline{2-2}
        ~                                     & \cellcolor{green!25}System tunability                    \\ \cline{2-2}
        ~                                     & \cellcolor{green!25}System testing                       \\ \hline \hline
        MIDI                                  & \cellcolor{green!25}USB MIDI and physical port           \\ \hline \hline
        Lighting Control                      & \cellcolor{green!25}Lighting control over MIDI           \\ \hline
    \end{tabular}

\end{table}

\begin{table}[!ht]
    \caption{Project stretch-goal list}\label{tab:scope_stretch}
    \centering
        \begin{tabular}{|l|c|}
        \hline
        Sensors                               & \cellcolor{orange!25}Wearable sensor design             \\ \hline \hline
        \multirow{3}{7em}{MIDI}               & \cellcolor{orange!25}Compliant MIDI implementation      \\ \cline{2-2}
        ~                                     & \cellcolor{orange!25}MIDI timecode quantisation         \\ \cline{2-2}
        ~                                     & \cellcolor{orange!25}MIDI output timecode               \\ \hline \hline
        \multirow{4}{7em}{System Integration} & \cellcolor{orange!25}Performance application testing    \\ \cline{2-2}
        ~                                     & \cellcolor{orange!25}Bi-directional communication       \\ \cline{2-2}
        ~                                     & \cellcolor{orange!25}Real-time system tunability        \\ \cline{2-2}
        ~                                     & \cellcolor{orange!25}PCB housing design for main board  \\ \hline
    \end{tabular}

\end{table}

\begin{table}[!ht]
    \caption{Project out-of-scope list}\label{tab:scope_out}
    \centering
        \begin{tabular}{|l|c|}
        \hline
        Sensors                               & \cellcolor{red!25}Sensors as independent systems        \\ \hline \hline
        \multirow{2}{7em}{MIDI}               & \cellcolor{red!25}Wireless MIDI                         \\ \cline{2-2}
        ~                                     & \cellcolor{red!25}MIDI threshold points                 \\ \hline \hline
        Lighting Control                      & \cellcolor{red!25}Lighting controller input support     \\ \hline \hline
        \multirow{2}{7em}{System Integration} & \cellcolor{red!25}Cableless system                      \\ \cline{2-2}
        ~                                     & \cellcolor{red!25}User interface for configuration      \\ \hline
    \end{tabular}

\end{table}

\chapter{Introduction}
This project proposes the use of biosignals, such as heart rate and muscle movement, to augment performances by generating correlated music and lighting.
For instance, a person could perform a movement routine and have the music of that routine be generated in response to their movement,
as opposed to learning a routine based on a preexisting piece of musical composition.
Similarly, lighting could be used to enhance a performance by allowing audiences to have a visual representation of the inner working of a performer's body,
and how that changes based on the state of the performance and the response of the audience.
This opens up room for future exploration into how performance can change when specific movements have additional audio and visual elements,
and how different movements may be endowed with novel meaning from these additions.

\section{Background}
A biosignal is a form of communication between biological systems~\cite{semmlow:2018},
they are used in the body to detect various biological events such as muscle contractions and heartbeats~\cite{escabí:2012}.
These signals can be detected using various types of sensors, including electric, mechanical, acoustic, and infrared sensors~\cite{kaniusas:2012}.

Music has been a form of human expression for over 40,000 years~\cite{killin:2018}.
In recent times in western musical expression, the creation of music has relied on the skill and dexterity of artists
who have dedicated years to practicing in order to become proficient.
This has presented accessibility challenges for individuals who may be unable to physically perform such actions
or to those who do not have the time required to learn.
This project offers a solution to this challenge by providing a platform for creating music that can be accessible to everyone.
Additionally, this project allows for multifaceted performances due to the lack of physical restrictions on performers
during the dynamic creation of music.

This project is also beneficial to current artists as the addition of lighting control allows for more engaging performances.
Traditionally, lighting control is operated manually by a skilled lighting technician or automatically triggered by sound.
This project allows the lighting to be controlled directly by the performer, which could allow for much more compelling lighting setups.

\section{Aims}
The aims of this project are to:

\begin{itemize}
        \item Control music and lighting in real-time from biosignals.
        \item Operate effectively in live performance spaces.
        \item Be wearable for an extended period of time without causing physical distress.
\end{itemize}

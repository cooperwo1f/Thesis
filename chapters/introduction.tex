\chapter{Introduction}
This project proposes the use of biological signals, such as heart rate or muscle contractions, to generate musical signals.
The project aims to provide a standard Musical Instrument Digital Interface (MIDI) for artists and performers to use.
Additionally, the project aims to implement lighting control using the same biological signals to provide a novel interface for controlling lights.

\section{Background}
Music has been a form of human expression for over 40,000 years~\cite{killin:2018}.
Throughout this time, the creation of music has relied on the skill and dexterity of artists
who have dedicated years to practicing in order to become proficient.
This has presented accessibility challenges for individuals who may be unable to physically perform such actions
or to those who do not have the time required to learn.
This project offers a solution to this challenge by providing a platform for creating music that can be accessible to everyone.
Additionally, this project allows for multifaceted performances due to the lack of physical restrictions on performers
during the dynamic creation of music.

This project is also beneficial to current artists as the addition of lighting control allows for more engaging performances.
Previously, lighting control has been done manually by a skilled lighting technician or automatically triggered by sound.
This project allows the lighting to be controlled directly by the performer, which could allow for much more compelling lighting setups.


A biosignal is a form of communication between biological systems~\cite{semmlow:2018},
they are used in the body to detect various biological events such as muscle contractions and heartbeats~\cite{escabí:2012}.
These signals can be detected using various types of sensors, including electric, mechanical, acoustic, and infrared sensors~\cite{kaniusas:2012}.

\section{Aims}
These are the aims

I need something like an objective at the beginning though since that should be my metric
for what to research in the literature review.

Maybe aims -> literature review -> objectives -> scope?

This is what we want, this is what already exists, this is what we want to do differently,
this is what we will do?


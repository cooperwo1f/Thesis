\chapter{Lighting Controller}
\section{Prototyping Fixture}
Small number of LEDs to verify correct control.
8 NeoPixel LEDs with 4 `dummy' DMX channels each.
DMX channels are intensity, red channel, green channel, blue channel.
Total of 32 channels (\(4 \times 8\))

\subsection{DMX}
Fixture uses Arduino along with
\href{https://github.com/mathertel/DMXSerial}DMXSerial library.
Receives DMX frames using interrupts
and stores them into array for processing.
DMX frames are 512 bytes wide.
A frame consists of the entire DMX `universe' of channels.
Individual channels are never written,
instead the entire `universe' is updated with each frame.
Updates happen continually at a known rate.
This way, fixture are aware if they lose connection to the controller,
since they stop receiving frames.

\subsection{NeoPixels}
LEDs controlled using \href{https://github.com/adafruit/Adafruit_NeoPixel}NeoPixel library.
LEDs can be colored independently using red, green, and blue values.
Each color is represented using a byte.

\subsection{Integration}
Both the DMX channel and color values are represented using bytes.
No additional scaling is required.

Brightness needs to be mapped using
\begin{lstlisting}[language=C]
  (intensity * color) >> 8;
\end{lstlisting}

In this case bit shifting by 8 is equivalent to dividing by 255.
This means that at maximum intensity the result of this calculation is the color value.
While at the minimum intensity the calculation becomes 0.

The values are sent over serial using Base64.
This is so that the carrige return line can be used to mark the end of the incoming values.
This was because the Arduino could become out of sync with what was being sent.
If this happened there is no way of getting back in sync,
because any special character could be interpreted as a regular value.
This is why Base64 is necessary, because it allows for special characters that are seperate
from the designated regular character set.
The values are sent as single decimal digits.
So when they are received they must be decoded back into bytes.

The value of each LED is then encoded across 4 bytes.
So the main program loops, incrementing by 4 each time,
and extracts the discrete bytes, setting the base address of the LED to the derived color.
Once this is done for all the LEDs, they are updated using the show() function.

\section{OpenDMX Controller}
When using expensive hardware in professional settings, it is important to use compliant hardware.
While the prototyping fixture could be used for system verification,
it was decided that existing hardware would be purchased for lighting control.
\textbf{DECISION MATRIX HERE (MIGHT ALREADY HAVE ONE IN MY RESULTS POWER POINT?)}

\subsection{Interface}
The driver provided by the manufacturer was written in C\#.
Existing code for the project is written in MATLAB.
We needed some way of controlling the device from MATLAB to utilise previous code.

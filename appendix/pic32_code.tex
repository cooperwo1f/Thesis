\chapter{PIC32 Code}\label{appendix:pic32_code}

\begin{lstlisting}[language=C]

// PIC32MX775F512H Configuration Bit Settings

// 'C' source line config statements

// DEVCFG3
// USERID = No Setting
#pragma config FSRSSEL = PRIORITY_7     // SRS Select (SRS Priority 7)
#pragma config FMIIEN = OFF             // Ethernet RMII/MII Enable (RMII Enabled)
#pragma config FETHIO = OFF             // Ethernet I/O Pin Select (Alternate Ethernet I/O)
#pragma config FCANIO = OFF             // CAN I/O Pin Select (Alternate CAN I/O)
#pragma config FUSBIDIO = ON            // USB USID Selection (Controlled by the USB Module)
#pragma config FVBUSONIO = ON           // USB VBUS ON Selection (Controlled by USB Module)

// DEVCFG2
#pragma config FPLLIDIV = DIV_10        // PLL Input Divider (10x Divider)
#pragma config FPLLMUL = MUL_16         // PLL Multiplier (16x Multiplier)
#pragma config UPLLIDIV = DIV_6         // USB PLL Input Divider (6x Divider)
#pragma config UPLLEN = ON              // USB PLL Enable (Enabled)
#pragma config FPLLODIV = DIV_8         // System PLL Output Clock Divider (PLL Divide by 8)

// DEVCFG1
#pragma config FNOSC = PRIPLL           // Oscillator Selection Bits (Primary Osc w/PLL (XT+,HS+,EC+PLL))
#pragma config FSOSCEN = OFF            // Secondary Oscillator Enable (Disabled)
#pragma config IESO = OFF               // Internal/External Switch Over (Disabled)
#pragma config POSCMOD = HS             // Primary Oscillator Configuration (HS osc mode)
#pragma config OSCIOFNC = OFF           // CLKO Output Signal Active on the OSCO Pin (Disabled)
#pragma config FPBDIV = DIV_1           // Peripheral Clock Divisor (Pb_Clk is Sys_Clk/1)
#pragma config FCKSM = CSDCMD           // Clock Switching and Monitor Selection (Clock Switch Disable, FSCM Disabled)
#pragma config WDTPS = PS1048576        // Watchdog Timer Postscaler (1:1048576)
#pragma config FWDTEN = OFF             // Watchdog Timer Enable (WDT Disabled (SWDTEN Bit Controls))

// DEVCFG0
#pragma config DEBUG = OFF              // Background Debugger Enable (Debugger is disabled)
#pragma config ICESEL = ICS_PGx1        // ICE/ICD Comm Channel Select (ICE EMUC1/EMUD1 pins shared with PGC1/PGD1)
#pragma config PWP = OFF                // Program Flash Write Protect (Disable)
#pragma config BWP = OFF                // Boot Flash Write Protect bit (Protection Disabled)
#pragma config CP = OFF                 // Code Protect (Protection Disabled)

// #pragma config statements should precede project file includes.
// Use project enums instead of #define for ON and OFF.

#include <xc.h>
#include "user.h"

int main (void) {
    init();

    while (1) {
        run();
    }

    // Should never reach this
    return 0;
}

\end{lstlisting}


\begin{lstlisting}[language=C]

#ifndef _SINE_H
#define _SINE_H

#ifdef __cplusplus
extern "C" {
#endif

void init(void);
void run(void);

#ifdef __cplusplus
}
#endif

#endif /* _SINE_H */
#ifndef _SINE_H
#define _SINE_H

#ifdef __cplusplus
extern "C" {
#endif

void init(void);
void run(void);

#ifdef __cplusplus
}
#endif

#endif /* _SINE_H */

\end{lstlisting}

\begin{lstlisting}[language=C]

#include <xc.h>
#include <stdint.h>
#include <string.h>

#include "user.h"
#include "util.h"
#include "ESP32.h"
#include "ADS1294R.h"

struct {
    uint32_t ECG:16;
    uint32_t RSP:16;
    uint32_t EMG:16;
    uint32_t BPM:16;
} packet;

ChannelData ch;

void init() {
//    ADC_init();
    ESP32_init();
    ADS1294R_init();
}

void run() {
//    if (data_ready()) {
//        read_data(&ch);
//        debug(
//            "HEADER: 0x%06X \n"
//            "CH1: %u \n"
//            "CH2: %u \n"
//            "CH3: %u \n"
//            "CH4: %u \n",
//            ch.HEAD, ch.CH1, ch.CH2, ch.CH3, ch.CH4
//        );
//    }
    debug("A");
//    delay(500);
}
\end{lstlisting}

\begin{lstlisting}[language=C]

#ifndef _UTIL_H_
#define _UTIL_H_

#include <xc.h>

// Print 8-bit binary number using printf
// usage: debug("NUM: "BYTE_TO_BINARY_PATTERN, BYTE_TO_BINARY(binary_number));
#define BYTE_TO_BINARY_PATTERN "%c%c%c%c%c%c%c%c"
#define BYTE_TO_BINARY(byte)  \
  ((byte) & 0x80 ? '1' : '0'), \
  ((byte) & 0x40 ? '1' : '0'), \
  ((byte) & 0x20 ? '1' : '0'), \
  ((byte) & 0x10 ? '1' : '0'), \
  ((byte) & 0x08 ? '1' : '0'), \
  ((byte) & 0x04 ? '1' : '0'), \
  ((byte) & 0x02 ? '1' : '0'), \
  ((byte) & 0x01 ? '1' : '0')

// SYS_FREQ = CRYSTAL_FREQ / 10 * 16 / 8
//#define SYS_FREQ 5000000

// May not be exactly actually since scope says different but close enough
#define DELAY_CONST 2.5

void delay_us(unsigned int us) {
    // Convert microseconds us into how many clock ticks it will take
    // Doing this causes an overflow I think... better to precalc and put in magic number
//    us *= SYS_FREQ / 1000000 / 2;   // Core Timer updates every 2 ticks
    us *= DELAY_CONST;
    _CP0_SET_COUNT(0);              // Set Core Timer count to 0
    while (us > _CP0_GET_COUNT());  // Wait until Core Timer count reaches the number we calculated earlier
}

void delay_ms(int ms) {
    delay_us(ms * 1000);
}

void delay(int ms) {
    delay_ms(ms);
}

void ADC_init() {
    AD1CON1bits.ADSIDL = 0;
    AD1CON1bits.SIDL = 0;
    AD1CON1bits.ASAM = 1;   // auto sampling
    AD1CON1bits.CLRASAM = 0; // overwrite buffer
    AD1CON1bits.FORM = 0b000; // integer 16-bit output
    AD1CON1bits.SSRC = 0b111; // auto convert
    AD1CON1bits.ADON = 1;
    AD1CON1bits.ON = 1;
    AD1CON1bits.SAMP = 1;

    AD1CHSbits.CH0SA = 0b1111;
    AD1CHSbits.CH0NA = 0;
    AD1CHSbits.CH0SB = 0b0000;
    AD1CHSbits.CH0NB = 0;
}
#endif // _UTIL_H_
\end{lstlisting}

\begin{lstlisting}[language=C]

#ifndef _ESP32_H_
#define _ESP32_H_

#include <xc.h>
#include <stdint.h>
#include <stdio.h>
#include <stdarg.h>
#include <string.h>

uint32_t ESP32_SPI_write(uint32_t data) {
    // Low-level SPI driver
    SPI2BUF = data;                 // Place data we want to send in SPI buffer
    while(!SPI2STATbits.SPITBE);    // Wait until sent status bit is cleared
    uint32_t read = SPI2BUF;        // Read data from buffer to clear it

    delay_us(5000);
    return read;
}

void ESP32_SPI_write_4byte(uint8_t b1, uint8_t b2, uint8_t b3, uint8_t b4) {
    uint32_t word = ((uint32_t)b1 << 24) | ((uint32_t)b2 << 16) | ((uint32_t)b3 << 8) | (uint32_t)b4;
    ESP32_SPI_write(word);
}

void ESP32_SPI_write_byte(uint8_t data) {
    ESP32_SPI_write_4byte(data, 0, 0, 0);
}

void ESP32_SPI_write_array(uint8_t *array, size_t len) {
    for (size_t i = 0; i < len; i++) {
        ESP32_SPI_write_byte(array[i]);
    }
}

void write_packet(uint8_t* buf, size_t len) {
    uint8_t mod_table[] = {0, 2, 1};
    char encoding_table[] = {   'A', 'B', 'C', 'D', 'E', 'F', 'G', 'H',
                                'I', 'J', 'K', 'L', 'M', 'N', 'O', 'P',
                                'Q', 'R', 'S', 'T', 'U', 'V', 'W', 'X',
                                'Y', 'Z', 'a', 'b', 'c', 'd', 'e', 'f',
                                'g', 'h', 'i', 'j', 'k', 'l', 'm', 'n',
                                'o', 'p', 'q', 'r', 's', 't', 'u', 'v',
                                'w', 'x', 'y', 'z', '0', '1', '2', '3',
                                '4', '5', '6', '7', '8', '9', '+', '/'
    };

    size_t output_length = 4 * ((len + 2) / 3);
    char encoded_data[output_length];

    for (int i = 0, j = 0; i < len;) {
        uint32_t octet_a = i < len ? buf[i++] : 0;
        uint32_t octet_b = i < len ? buf[i++] : 0;
        uint32_t octet_c = i < len ? buf[i++] : 0;

        uint32_t triple = (octet_a << 0x10) + (octet_b << 0x08) + octet_c;

        encoded_data[j++] = encoding_table[(triple >> 3 * 6) & 0x3F];
        encoded_data[j++] = encoding_table[(triple >> 2 * 6) & 0x3F];
        encoded_data[j++] = encoding_table[(triple >> 1 * 6) & 0x3F];
        encoded_data[j++] = encoding_table[(triple >> 0 * 6) & 0x3F];
    }

    for (int i = 0; i < mod_table[len % 3]; i++) {
        encoded_data[output_length - 1 - i] = '=';
    }

    ESP32_SPI_write_array(encoded_data, output_length);
    ESP32_SPI_write_byte('\n');
}

void debug(const char *fmt, ...) {
    va_list args;
    char str[1024];

    va_start(args, fmt);
    vsprintf(str, fmt, args);
    va_end(args);

    write_packet(str, strlen(str));
}

void ESP32_IO_init() {
    TRISBbits.TRISB2 = 0;       // Set ESP32 EN pin as output
    PORTBbits.RB2 = 1;          // Set ESP32 EN pin high
}

void ESP32_SPI_init() {
    SPI2CONbits.ON = 0;         // Turn off SPI2 before configuring
    SPI2CONbits.FRMEN = 0;      // Framed SPI Support (SS pin used)
    SPI2CONbits.MSSEN = 1;      // Slave Select Enable (SS driven during transmission)
    SPI2CONbits.ENHBUF = 0;     // Enhanced Buffer Enable (disable enhanced buffer)
    SPI2CONbits.SIDL = 1;       // Stop in Idle Mode
    SPI2CONbits.DISSDO = 0;     // Disable SDOx (pin is controlled by this module)
    SPI2CONbits.MODE32 = 1;     // Use 32-bit mode
    SPI2CONbits.MODE16 = 0;     // Do not use 16-bit mode
    SPI2CONbits.SMP = 0;        // Input data is sampled at the end of the clock signal
    SPI2CONbits.CKE = 1;        // Data is shifted out/in on transition from idle (high) state to active (low) state
    SPI2CONbits.SSEN = 1;       // Slave Select Enable (SS pin used by module)
    SPI2CONbits.CKP = 0;        // Clock Polarity Select (clock signal is active low, idle state is high)
    SPI2CONbits.MSTEN = 1;      // Master Mode Enable
    SPI2CONbits.STXISEL = 0b01; // SPI Transmit Buffer Empty Interrupt Mode (generated when the buffer is completely empty)
    SPI2CONbits.SRXISEL = 0b11; // SPI Receive Buffer Full Interrupt Mode (generated when the buffer is full)
    SPI2BRG = 50;
    SPI2CONbits.ON = 1;         // Configuration is done, turn on SPI2 peripheral
}

void ESP32_init() {
    ESP32_IO_init();
    ESP32_SPI_init();
}

#endif /* _ESP32_H_ */
\end{lstlisting}

\begin{lstlisting}[language=C]

#ifndef _ADS1294R_H_
#define _ADS1294R_H_

#include <xc.h>
#include <stdint.h>

#include "util.h"
#include "ESP32.h"


/* Pin Mapping */

// Test points
#define TP6_PIN PORTDbits.RD4
#define TP7_PIN PORTDbits.RD5
#define TP8_PIN PORTDbits.RD6

// Controls
#define DRDY_PIN PORTDbits.RD7
#define CLKSEL_PIN PORTDbits.RD8
#define CS_PIN PORTDbits.RD9
#define START_PIN PORTDbits.RD10


/* Register Addresses */

// Device settings (READ-ONLY)
#define ID          0x00

// Global Settings across channels
#define CONFIG1     0x01
#define CONFIG2     0x02
#define CONFIG3     0x03
#define LOFF        0x04

// Channel-specific settings
#define CH1SET      0x05
#define CH2SET      0x06
#define CH3SET      0x07
#define CH4SET      0x08
#define RLD_SENSP   0x0D
#define RLD_SENSN   0x0E
#define LOFF_SENSP  0x0F
#define LOFF_SENSN  0x10
#define LOFF_FLIP   0x11

// Lead-off status registers (READ-ONLY)
#define LOFF_STATP  0x12
#define LOFF_STATN  0x13

// GPIO and other registers
#define GPIO        0x14
#define PACE        0x15
#define RESP        0x16
#define CONFIG4     0x17
#define WCT1        0x18
#define WCT2        0x19


/* SPI Command Definitions */

// System commands
#define WAKEUP      0x02
#define STANDBY     0x04
#define RESET       0x06
#define START       0x08
#define STOP        0x0A

// Data read commands
#define RDATAC      0x10
#define SDATAC      0x11
#define RDATA       0x12


/* Chip info */

// Channel definitions
#define NUMBER_OF_CHANNELS 4
#define BYTES_PER_CHANNEL 3
#define BYTES_TO_READ (NUMBER_OF_CHANNELS * BYTES_PER_CHANNEL)

#define CS_DELAY 0


/* Channel data struct */
typedef struct {
    uint32_t HEAD:24;
    uint32_t CH1:24;
    uint32_t CH2:24;
    uint32_t CH3:24;
    uint32_t CH4:24;
} ChannelData;

#define THREE_BYTE(B1, B2, B3) ((B1 << 16) | (B2 << 8) | B3)


/* Low-level driver */

uint8_t ADS1294R_write(uint8_t data) {
    // Low-level SPI driver
    SPI3BUF = (uint32_t)data;           // Place data we want to send in SPI buffer
    while(!SPI3STATbits.SPITBE);        // Wait until sent status bit is cleared
    return (uint8_t)SPI3BUF;            // Read data from buffer to clear it
}

uint8_t ADS1294R_read() {
    return ADS1294R_write(0x00);
}

void write_cmd(uint8_t cmd) {
    CS_PIN = 0;
    ADS1294R_write(cmd);
    CS_PIN = 1;
}

/* Register drivers */

uint8_t read_register(uint8_t reg) {
    static uint8_t read_register_cmd = 0x20;
    static uint8_t read_register_mask = 0x1F;

    uint8_t first_byte = read_register_cmd | (reg & read_register_mask);
    uint8_t second_byte = 0x00; // only ever read a single register

    CS_PIN = 0;
    ADS1294R_write(first_byte);
    ADS1294R_write(second_byte);
    ADS1294R_read();
    uint8_t ret = ADS1294R_read();
    CS_PIN = 1;

    return ret;
}

void write_register(uint8_t reg, uint8_t data) {
    static uint8_t write_register_cmd = 0x40;
    static uint8_t write_register_mask = 0x1F;

    uint8_t first_byte = write_register_cmd | (reg & write_register_mask);
    uint8_t second_byte = 0x00; // only ever write a single register

    CS_PIN = 0;
    ADS1294R_write(first_byte);
    ADS1294R_write(second_byte);
    ADS1294R_write(data);
    CS_PIN = 1;
}


/* ADS1298RADS1294R init */

void ADS1294R_GPIO_init() {
    // Not sure if any of these work...
    TRISDbits.TRISD4 = 0;       // TP6 as output    -  Pin 52
    TRISDbits.TRISD5 = 0;       // TP7 as output    -  Pin 53
    TRISDbits.TRISD6 = 0;       // TP8 as output    -  Pin 54

    TRISDbits.TRISD7 = 1;       // nDRDY as input   -  Pin 55
    TRISDbits.TRISD8 = 0;       // CLKSEL as output -  Pin 42
    TRISDbits.TRISD9 = 0;       // CS as output     -  Pin 43
    TRISDbits.TRISD10 = 0;      // START as output  -  Pin 44

    TP6_PIN = 0;
    TP7_PIN = 0;
    TP8_PIN = 0;

    CLKSEL_PIN = 0;
    CS_PIN = 1;
    START_PIN = 0;
}

void ADS1294R_SPI_init() {
    SPI3CONbits.ON = 0;         // Turn off SPI2 before configuring
    SPI3CONbits.FRMEN = 0;      // Framed SPI Support (SS pin used)
    SPI3CONbits.MSSEN = 0;      // Slave Select Enable (SS driven during transmission)
    SPI3CONbits.ENHBUF = 0;     // Enhanced Buffer Enable (disable enhanced buffer)
    SPI3CONbits.SIDL = 1;       // Stop in Idle Mode
    SPI3CONbits.DISSDO = 0;     // Disable SDOx (pin is controlled by this module)
    SPI3CONbits.MODE32 = 0;     // Do not use 32-bit mode (8-bit mode)
    SPI3CONbits.MODE16 = 0;     // Do not use 16-bit mode (8-bit mode)

    // SMP = 1; data sampled at end of output time... SMP = 0; data sampled at middle of output time
    SPI3CONbits.SMP = 1;

    // CKE = 1; transition from active to idle... CKE = 0; transition from idle to active
    SPI3CONbits.CKE = 0;

    // CKP = 1; high is idle, low is active... CKP = 0; low is idle, high is active
    SPI3CONbits.CKP = 0;

    SPI3CONbits.SSEN = 0;       // Slave Select Enable (SS pin used by module)
    SPI3CONbits.MSTEN = 1;      // Master Mode Enable
    SPI3CONbits.STXISEL = 0b01; // SPI Transmit Buffer Empty Interrupt Mode (generated when the buffer is completely empty)
    SPI3CONbits.SRXISEL = 0b11; // SPI Receive Buffer Full Interrupt Mode (generated when the buffer is full)

    // SCLK period > 70ns
    // 70ns ~= 14.3MHz
    // F_SCK = 14MHz

    // Library uses 4MHz

    // BRG = (F_PB / 2 * F_SCK) - 1
    // BRG = 1.86
    // BRG >= 2

    SPI3BRG = 4;
    SPI3CONbits.ON = 1;         // Configuration is done, turn on SPI3 peripheral
}


/* Public Functions */

void ADS1294R_init() {
    ADS1294R_GPIO_init();
    ADS1294R_SPI_init();

    // Set CLKSEL pin = 1
    CLKSEL_PIN = 1;
    delay(1);

    write_cmd(RESET);
    delay(1);

    // Send Stop Data Continuous command
    write_cmd(SDATAC);
    delay(1);

//    write_register(GPIO, 0b11110000);

    // Write config registers
    write_register(CONFIG1, 0x86);  // 500 samples/s
    delay(1);
    write_register(CONFIG2, 0x00);  // Test signals disabled
    delay(1);
    write_register(CONFIG3, 0xC0);  // Enable internal reference buffer, no RLD
    delay(1);

    // Send Read Data Continuous command
    START_PIN = 1;
    delay(1);
    write_cmd(START);
    delay(1);
    write_cmd(RDATAC);
    delay(1);
}

void read_data(ChannelData* ch) {
    CS_PIN = 0;

    ADS1294R_read();    // read once to clear out previous buffer
//    ch->HEAD = THREE_BYTE(ADS1294R_read(), ADS1294R_read(), ADS1294R_read());
//    ch->CH1 = THREE_BYTE(ADS1294R_read(), ADS1294R_read(), ADS1294R_read());
//    ch->CH2 = THREE_BYTE(ADS1294R_read(), ADS1294R_read(), ADS1294R_read());
//    ch->CH3 = THREE_BYTE(ADS1294R_read(), ADS1294R_read(), ADS1294R_read());
//    ch->CH4 = THREE_BYTE(ADS1294R_read(), ADS1294R_read(), ADS1294R_read());

    uint8_t h1 = ADS1294R_read();
    uint8_t h2 = ADS1294R_read();
    uint8_t h3 = ADS1294R_read();

    ch->HEAD = (h1 << 16) | (h2 << 8) | h3;

    for (uint8_t i = 0; i < BYTES_TO_READ; i++) {
        ADS1294R_read();
    }

    CS_PIN = 1;
}

uint8_t data_ready() {
    return DRDY_PIN == 0;
}

#endif /* _ADS1294R_H_ */
\end{lstlisting}
